\documentclass[11pt,a4paper]{article}
\usepackage{amsmath}
\usepackage{amssymb}
\usepackage{listings}
\usepackage{xcolor}
\usepackage{hyperref}

\lstset{
    language=Python,
    basicstyle=\ttfamily\small,
    keywordstyle=\color{blue},
    commentstyle=\color{gray},
    stringstyle=\color{red},
    showstringspaces=false,
    breaklines=true,
    frame=single,
    numbers=left,
    numberstyle=\tiny\color{gray}
}

\title{ Function System }
\author{MathHook CAS}
\date{\today}

\begin{document}

\maketitle

\begin{abstract}
MathHook provides a comprehensive mathematical function system with intelligent evaluation,
symbolic manipulation, and educational explanations. Functions are first-class expressions
supporting exact symbolic computation and high-performance numerical evaluation through
a modular intelligence architecture.

\end{abstract}


\section{Mathematical Definition}

\begin{equation}
Functions in MathHook follow standard mathematical definitions:

**Trigonometric**: $$\sin(x), \cos(x), \tan(x)$$ with periodicity $$2\pi$$

**Exponential/Logarithmic**: $$e^x, \ln(x), \log_b(x) = \frac{\ln(x)}{\ln(b)}$$

**Special Functions**: $$\Gamma(n) = \int_0^{\infty} t^{n-1} e^{-t} dt$$

\end{equation}



\section{Introduction}

[Full markdown content preserved from functions.md - truncated in this example for brevity]





\section{Examples}


\subsection{ Creating Functions with Macros }

Using function! and expr! macros for ergonomic function creation

\begin{lstlisting}
from mathhook import symbol, function, expr

x = symbol('x')

# Single argument functions
sine = function('sin', x)
cosine = function('cos', x)

# Multi-argument functions
log_base = function('log', x, 10)

# Using expr
trig_identity = expr('sin(x)^2 + cos(x)^2')
assert trig_identity.simplify() == 1

\end{lstlisting}




\subsection{ Trigonometric Functions with Exact Values }

Automatic recognition of exact trigonometric values at special angles

\begin{lstlisting}
from mathhook import expr

# Exact values recognized
assert expr('sin(0)') == 0
assert expr('sin(pi/6)') == expr('1/2')
assert expr('sin(pi/4)') == expr('sqrt(2)/2')
assert expr('sin(pi/2)') == 1

assert expr('cos(0)') == 1
assert expr('cos(pi/3)') == expr('1/2')
assert expr('cos(pi/2)') == 0

\end{lstlisting}




\subsection{ Logarithm and Exponential Identities }

Automatic application of logarithm laws and exponential identities

\begin{lstlisting}
from mathhook import symbol, expr

a = symbol('a')
b = symbol('b')
n = symbol('n')

# Logarithm laws
assert expr('ln(a*b)').expand() == expr('ln(a) + ln(b)')
assert expr('ln(a/b)').expand() == expr('ln(a) - ln(b)')
assert expr('ln(a^n)').expand() == expr('n*ln(a)')

# Exponential identities
assert expr('e^(ln(a))').simplify() == a
assert expr('ln(e^a)').simplify() == a

\end{lstlisting}




\subsection{ Function Derivatives (Automatic Chain Rule) }

Functions know their derivatives with automatic chain rule application

\begin{lstlisting}
from mathhook import symbol, expr

x = symbol('x')

# Basic derivatives
assert expr('sin(x)').derivative(x) == expr('cos(x)')
assert expr('cos(x)').derivative(x) == expr('-sin(x)')
assert expr('exp(x)').derivative(x) == expr('exp(x)')
assert expr('ln(x)').derivative(x) == expr('1/x')

# Chain rule automatic
f = expr('sin(x^2)')
assert f.derivative(x) == expr('2*x*cos(x^2)')

\end{lstlisting}




\subsection{ Special Functions (Gamma and Bessel) }

Advanced special functions for scientific and engineering applications

\begin{lstlisting}
from mathhook import expr, symbol

# Gamma function
assert expr('gamma(1)') == 1
assert expr('gamma(5)') == 24
assert expr('gamma(1/2)') == expr('sqrt(pi)')

# Bessel functions
x = symbol('x')
bessel_j0 = expr('bessel_j(0, x)')
bessel_y0 = expr('bessel_y(0, x)')

\end{lstlisting}




\subsection{ Real-World Physics Application }

Damped harmonic oscillator using exponential and trigonometric functions

\begin{lstlisting}
from mathhook import symbol, expr

# Damped harmonic motion
t = symbol('t')
position = expr('e^(-0.1*t) * cos(2*pi*t)')
velocity = position.derivative(t)
acceleration = velocity.derivative(t)

# Differential equation verification
gamma = 0.1
omega = expr('2*pi')
lhs = expr(f'acceleration + 2*{gamma}*velocity + omega^2*position')
# Should simplify to 0

\end{lstlisting}







\end{document}
