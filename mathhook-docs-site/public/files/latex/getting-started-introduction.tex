\documentclass[11pt,a4paper]{article}
\usepackage{amsmath}
\usepackage{amssymb}
\usepackage{listings}
\usepackage{xcolor}
\usepackage{hyperref}

\lstset{
    language=Python,
    basicstyle=\ttfamily\small,
    keywordstyle=\color{blue},
    commentstyle=\color{gray},
    stringstyle=\color{red},
    showstringspaces=false,
    breaklines=true,
    frame=single,
    numbers=left,
    numberstyle=\tiny\color{gray}
}

\title{ Introduction to MathHook }
\author{MathHook CAS}
\date{\today}

\begin{document}

\maketitle

\begin{abstract}
MathHook is a high-performance educational computer algebra system (CAS) written in Rust,
designed to combine mathematical correctness with exceptional performance.

\end{abstract}




\section{Introduction}

# Introduction

Welcome to the MathHook documentation! MathHook is a high-performance educational computer algebra system (CAS) written in Rust, designed to combine mathematical correctness with exceptional performance.

## What is MathHook?

MathHook is a symbolic mathematics engine that can:

- **Parse** mathematical expressions from multiple formats (LaTeX, Wolfram Language, standard notation)
- **Simplify** algebraic expressions using canonical forms and mathematical identities
- **Differentiate** and **integrate** expressions symbolically
- **Solve** equations and systems of equations
- **Manipulate** matrices with full linear algebra support
- **Explain** mathematical operations step-by-step for educational purposes

## Why MathHook?

### Performance-First Design

MathHook is built from the ground up for speed:

- **32-byte expression representation** fits perfectly in CPU cache lines
- **SIMD operations** for vectorized arithmetic (2-4x speedup)
- **Zero-copy parsing** directly constructs AST without intermediate allocations
- **Thread-safe immutable expressions** enable parallel processing
- **10-100x faster** than SymPy for common operations

### Mathematical Correctness

Every operation in MathHook is designed to be mathematically correct:

- Exact rational arithmetic (never loses precision)
- Proper domain handling (sqrt, log, division by zero)
- Canonical forms for reliable equality checking
- Validated against SymPy

### Educational Focus

MathHook provides step-by-step explanations for all mathematical operations, making it ideal for:

- Educational software
- Mathematics learning platforms
- Interactive mathematics tools
- Automated tutoring systems

### Multi-Language Support

MathHook provides first-class bindings for:

- **Rust** (native API with ergonomic macros)
- **Python** (via PyO3)
- **Node.js/TypeScript** (via NAPI-RS)
- **WebAssembly** (coming soon)

## Key Features

### Expression Building

Create mathematical expressions naturally using the `expr!` and `symbol!` macros.

### Symbolic Computation

Perform algebraic manipulations symbolically:
- Simplification
- Expansion
- Factorization

### Calculus Operations

Compute derivatives and integrals symbolically.

### Equation Solving

Solve equations and systems of equations.

### Matrix Operations

Full linear algebra support with symbolic and numeric matrices.

## Architecture

MathHook is organized as a multi-crate workspace:

- **mathhook-core**: Core mathematical engine (pure Rust)
- **mathhook**: High-level API with ergonomic macros
- **mathhook-python**: Python bindings
- **mathhook-node**: Node.js/TypeScript bindings
- **mathhook-benchmarks**: Performance benchmarking suite

## Design Principles

MathHook follows five core principles (in priority order):

1. **Mathematical Correctness First**: Every operation must be mathematically correct
2. **Performance**: Cache-friendly data structures, SIMD operations, parallel processing
3. **Ergonomic API**: Macros and operator overloading for natural expression
4. **Educational Value**: Step-by-step explanations for all operations
5. **Multi-Language**: First-class bindings for Python, Node.js, and WebAssembly





\section{Examples}


\subsection{ Expression Building }

Create mathematical expressions using macros

\begin{lstlisting}
from mathhook import symbol, expr

x = symbol('x')
expression = expr('x^2 + 2*x + 1')

\end{lstlisting}




\subsection{ Symbolic Computation }

Perform algebraic manipulations

\begin{lstlisting}
from mathhook import symbol, expr

x = symbol('x')
expression = expr('x^2 + 2*x + 1')

simplified = expression.simplify()
expanded = expression.expand()
factored = expression.factor()

\end{lstlisting}




\subsection{ Calculus Operations }

Compute derivatives and integrals

\begin{lstlisting}
from mathhook import symbol, expr

x = symbol('x')
expression = expr('x^2 + 2*x + 1')

derivative = expression.derivative(x)
integral = expression.integrate(x)

\end{lstlisting}







\end{document}
