\documentclass[11pt,a4paper]{article}
\usepackage{amsmath}
\usepackage{amssymb}
\usepackage{listings}
\usepackage{xcolor}
\usepackage{hyperref}

\lstset{
    language=Python,
    basicstyle=\ttfamily\small,
    keywordstyle=\color{blue},
    commentstyle=\color{gray},
    stringstyle=\color{red},
    showstringspaces=false,
    breaklines=true,
    frame=single,
    numbers=left,
    numberstyle=\tiny\color{gray}
}

\title{ Complete PDE Examples }
\author{MathHook CAS}
\date{\today}

\begin{document}

\maketitle

\begin{abstract}
Three complete, real-world examples demonstrating MathHook's PDE solving capabilities across heat, wave, and Laplace equations.
Each example includes full problem setup, mathematical formulation, MathHook implementation, and physical interpretation.

\end{abstract}


\section{Mathematical Definition}

\begin{equation}
**Example 1: Heat Diffusion**
$$\frac{\partial u}{\partial t} = \alpha \frac{\partial^2 u}{\partial x^2}$$

**Example 2: Wave Propagation**
$$\frac{\partial^2 u}{\partial t^2} = c^2 \frac{\partial^2 u}{\partial x^2}$$

**Example 3: Electrostatic Potential**
$$\frac{\partial^2 u}{\partial x^2} + \frac{\partial^2 u}{\partial y^2} = 0$$

\end{equation}



\section{Introduction}

# Complete PDE Examples

Three complete, real-world examples demonstrating MathHook's PDE solving capabilities.

## Example 1: Heat Diffusion in Steel Rod

**Physical Problem**: A 1-meter steel rod is initially heated to 100°C. Both ends are plunged into ice water (0°C). How does temperature evolve?

### Complete Solution

**STEP 1: Define Variables**

Temperature $u$, position $x$ (0 to 1 meter), time $t$

**STEP 2: Create PDE**

Heat equation: $\frac{\partial u}{\partial t} = \alpha \frac{\partial^2 u}{\partial x^2}$

**STEP 3: Material Properties**

Steel: $\alpha = k/(\rho c_p) \approx 1.3 \times 10^{-5}$ m²/s

**STEP 4: Boundary Conditions**

- $u(0,t) = 0°C$ (left end in ice water)
- $u(1,t) = 0°C$ (right end in ice water)

**STEP 5: Initial Condition**

$u(x,0) = 100°C$ (uniform initial temperature)

**STEP 6: Solve**

Using `HeatEquationSolver`

**STEP 7: Examine Solution**

Solution structure shows eigenvalues $\lambda_n = (n\pi/L)^2$ and exponential decay modes.

**Physical Interpretation**:
- Eigenvalues determine spatial modes
- Higher modes decay faster (∝ $n^2$)
- Temperature → 0°C as $t \to \infty$ (boundary temperature)

## Example 2: Vibrating Guitar String

**Physical Problem**: An E4 guitar string (0.65 m) is plucked 5 mm at the center and released. Describe the vibration.

### Complete Solution

**STEP 1: Define Variables**

Displacement $u$, position $x$ along string, time $t$

**STEP 2: Create PDE**

Wave equation: $\frac{\partial^2 u}{\partial t^2} = c^2 \frac{\partial^2 u}{\partial x^2}$

**STEP 3: Physical Parameters**

Steel E string: $T = 73.4$ N, $\rho = 3.75 \times 10^{-4}$ kg/m

Wave speed: $c = \sqrt{T/\rho} \approx 442$ m/s

**STEP 4: Boundary Conditions**

- $u(0,t) = 0$ (left end fixed)
- $u(L,t) = 0$ (right end fixed, $L = 0.65$ m)

**STEP 5: Initial Conditions**

- Initial position: triangular pluck at center (5 mm displacement)
- Initial velocity: released from rest ($\partial u/\partial t = 0$)

**STEP 6: Solve**

Using `WaveEquationSolver`

**STEP 7: Analyze Musical Properties**

**Musical Harmonics:**
- Fundamental: $f_1 = c/(2L) \approx 340$ Hz (close to E4 = 329.63 Hz)
- Overtones: $f_n = n f_1$
  - $f_2 = 680$ Hz (octave)
  - $f_3 = 1020$ Hz (octave + fifth)
  - $f_4 = 1360$ Hz (two octaves)

**Standing Wave Nodes:**
- Mode 1: nodes at $x = 0, 0.65$ m
- Mode 2: nodes at $x = 0, 0.325, 0.65$ m
- Mode 3: nodes at $x = 0, 0.217, 0.433, 0.65$ m

## Example 3: Electrostatic Potential in Rectangular Plate

**Physical Problem**: A 10 cm × 5 cm conducting plate has bottom/sides grounded (0 V) and top edge at 100 V. Find the potential distribution.

### Complete Solution

**STEP 1: Define Variables**

Electrostatic potential $u$, horizontal position $x$, vertical position $y$

**STEP 2: Create PDE**

Laplace equation: $\frac{\partial^2 u}{\partial x^2} + \frac{\partial^2 u}{\partial y^2} = 0$

**STEP 3: Boundary Conditions**

- $u(0,y) = 0$ V (left edge grounded)
- $u(a,y) = 0$ V (right edge grounded, $a = 0.1$ m)
- $u(x,0) = 0$ V (bottom edge grounded)
- $u(x,b) = 100$ V (top edge at fixed potential, $b = 0.05$ m)

**STEP 4: Solve**

Using `LaplaceEquationSolver`

**STEP 5: Examine Solution**

Solution structure shows:
- X-direction eigenvalues: $\lambda_n = (n\pi/a)^2$
- Hyperbolic sine functions in $y$-direction
- Smooth variation from 0 V to 100 V

**Physical Interpretation:**
- Potential varies smoothly from 0 V (bottom/sides) to 100 V (top)
- No local maxima/minima inside (maximum principle)
- Electric field $\mathbf{E} = -\nabla u$ points from high to low potential
- Field strongest near top edge (steepest gradient)

**Estimated potential at center (5 cm, 2.5 cm):**
$u(5, 2.5) \approx 48$ V (halfway between 0 V and 100 V)

## Common Pitfalls

### Pitfall 1: Expecting Numerical Coefficients

❌ WRONG: Coefficients are symbolic
```rust
for coeff in result.coefficients {
    let numerical_value = coeff.evaluate()?;  // ERROR: Can't evaluate A_1
}
```

✅ CORRECT: Acknowledge symbolic nature
```rust
for (n, coeff) in result.coefficients.iter().enumerate() {
    println!("Coefficient A_{} (symbolic): {}", n + 1, coeff);
}
```

### Pitfall 2: Using Non-Standard Variable Names

❌ MAY NOT CLASSIFY:
```rust
let r = symbol!(r);         // Radial
let theta = symbol!(theta); // Angular
```

✅ USE STANDARD NAMES:
```rust
let x = symbol!(x);
let y = symbol!(y);
let t = symbol!(t);
```

### Pitfall 3: Non-Homogeneous BCs Without Transformation

❌ UNSUPPORTED DIRECTLY:
```rust
let bc = BoundaryCondition::dirichlet(expr!(50), ...);  // Non-zero
```

✅ TRANSFORM FIRST:
1. Find steady-state $u_s(x)$ satisfying BCs
2. Solve for $v(x,t) = u(x,t) - u_s(x)$ with homogeneous BCs
3. Add back: $u(x,t) = v(x,t) + u_s(x)$

## Summary

**Three complete examples** demonstrate:
1. ✅ Heat equation: Thermal diffusion in steel
2. ✅ Wave equation: Musical string vibrations
3. ✅ Laplace equation: Electrostatic potential

**All examples show**:
- Correct eigenvalue computation
- Proper solution structure
- Physical interpretation
- Symbolic coefficient limitation

**Next steps**: Use these patterns for your own PDE problems, keeping limitations in mind (Dirichlet BCs only, symbolic coefficients).





\section{Examples}


\subsection{ Heat Diffusion in Steel Rod - Complete Implementation }

1-meter steel rod cooling from 100°C with ice water at ends. Full implementation with error handling.

\begin{lstlisting}
from mathhook import symbol, expr, Pde, BoundaryCondition, BoundaryLocation, InitialCondition, HeatEquationSolver

def solve_cooling_rod():
    # Define Variables
    u = symbol('u')
    x = symbol('x')
    t = symbol('t')

    # Create PDE
    equation = expr(u)
    pde = Pde(equation, u, [x, t])

    # Material Properties
    alpha = expr(0.000013)

    # Boundary Conditions
    bc_left = BoundaryCondition.dirichlet(
        expr(0),
        BoundaryLocation.simple(variable=x, value=expr(0))
    )
    bc_right = BoundaryCondition.dirichlet(
        expr(0),
        BoundaryLocation.simple(variable=x, value=expr(1))
    )

    # Initial Condition
    ic = InitialCondition.value(expr(100))

    # Solve
    solver = HeatEquationSolver()
    result = solver.solve_heat_equation_1d(pde, alpha, [bc_left, bc_right], ic)

    print(f"Solution: {result.solution}")
    print(f"Eigenvalues: {result.eigenvalues[:5]}")
    print(f"Coefficients: {result.coefficients[:5]}")

\end{lstlisting}




\subsection{ Vibrating Guitar String - Musical Analysis }

E4 guitar string with musical frequency analysis and standing wave nodes.

\begin{lstlisting}
from mathhook import symbol, expr, Pde, BoundaryCondition, BoundaryLocation, InitialCondition, WaveEquationSolver

def solve_vibrating_string():
    u = symbol('u')
    x = symbol('x')
    t = symbol('t')

    equation = expr(u)
    pde = Pde(equation, u, [x, t])

    c = expr(442)

    bc1 = BoundaryCondition.dirichlet(expr(0), BoundaryLocation.simple(variable=x, value=expr(0)))
    bc2 = BoundaryCondition.dirichlet(expr(0), BoundaryLocation.simple(variable=x, value=expr(0.65)))

    ic_position = InitialCondition.value(expr(0.005))
    ic_velocity = InitialCondition.derivative(expr(0))

    solver = WaveEquationSolver()
    result = solver.solve_wave_equation_1d(pde, c, [bc1, bc2], ic_position, ic_velocity)

    print(f"Solution: {result.solution}")

    # Musical frequencies
    L = 0.65
    c_val = 442.0
    for n in range(1, 6):
        f_n = n * c_val / (2.0 * L)
        print(f"f_{n} = {f_n:.2f} Hz (mode {n})")

\end{lstlisting}




\subsection{ Electrostatic Potential in Rectangular Plate }

10cm × 5cm plate with grounded sides and fixed potential top edge.

\begin{lstlisting}
from mathhook import symbol, expr, Pde, BoundaryCondition, BoundaryLocation, LaplaceEquationSolver

def solve_electrostatic_potential():
    u = symbol('u')
    x = symbol('x')
    y = symbol('y')

    equation = expr(u)
    pde = Pde(equation, u, [x, y])

    bc_left = BoundaryCondition.dirichlet(expr(0), BoundaryLocation.simple(variable=x, value=expr(0)))
    bc_right = BoundaryCondition.dirichlet(expr(0), BoundaryLocation.simple(variable=x, value=expr(0.1)))
    bc_bottom = BoundaryCondition.dirichlet(expr(0), BoundaryLocation.simple(variable=y, value=expr(0)))
    bc_top = BoundaryCondition.dirichlet(expr(100), BoundaryLocation.simple(variable=y, value=expr(0.05)))

    solver = LaplaceEquationSolver()
    result = solver.solve_laplace_equation_2d(pde, [bc_left, bc_right, bc_bottom, bc_top])

    print(f"Solution: {result.solution}")
    print(f"X-eigenvalues: {result.x_eigenvalues[:5]}")

\end{lstlisting}







\end{document}
