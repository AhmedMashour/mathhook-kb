\documentclass[11pt,a4paper]{article}
\usepackage{amsmath}
\usepackage{amssymb}
\usepackage{listings}
\usepackage{xcolor}
\usepackage{hyperref}

\lstset{
    language=Python,
    basicstyle=\ttfamily\small,
    keywordstyle=\color{blue},
    commentstyle=\color{gray},
    stringstyle=\color{red},
    showstringspaces=false,
    breaklines=true,
    frame=single,
    numbers=left,
    numberstyle=\tiny\color{gray}
}

\title{ Learning Paths }
\author{MathHook CAS}
\date{\today}

\begin{document}

\maketitle

\begin{abstract}
Choose your journey based on background and goals. Structured learning paths for
Python data scientists, Node.js developers, Rust programmers, mathematics educators,
and computational scientists with time estimates and outcomes.

\end{abstract}




\section{Introduction}

# Learning Paths

Choose your journey based on your background and goals. Each path is designed to
get you productive with MathHook as quickly as possible.

## Path 1: Python Data Scientist

**Background**: Familiar with NumPy, SymPy, pandas
**Goal**: Use MathHook for faster symbolic computation in Python
**Time to Productivity**: 1-2 hours

Learn Python API, performance comparison with SymPy, integration with data science
stack, and when to use MathHook vs SymPy.

## Path 2: Node.js/TypeScript Developer

**Background**: JavaScript/TypeScript web development
**Goal**: Add symbolic math to web applications
**Time to Productivity**: 2-3 hours

Learn Node.js bindings, LaTeX parsing for web forms, web framework integration,
and V8 optimization.

## Path 3: Rust Systems Programmer

**Background**: Rust experience, need high-performance CAS
**Goal**: Embed MathHook in Rust application or contribute to core
**Time to Productivity**: 4-6 hours to mastery

Learn architecture, memory layout, SIMD optimization, and custom extensions.

## Path 4: Mathematics Student/Educator

**Background**: Calculus, linear algebra, abstract algebra knowledge
**Goal**: Understand CAS internals, use for teaching, contribute
**Time to Productivity**: 8-12 hours to contribution-ready

Learn symbolic computation theory, algorithm implementation, and educational features.

## Path 5: Computational Scientist

**Background**: MATLAB, Julia, scientific computing
**Goal**: Fast symbolic preprocessing for numerical simulations
**Time to Productivity**: 3-4 hours

Learn symbolic matrix algebra, system solving, hybrid symbolic-numerical workflows,
and code generation.

## Common Themes

Essential concepts for all users:
- Expressions are immutable (safe for concurrent use)
- Canonical forms (x + y equals y + x)
- Exact vs approximate arithmetic (rationals vs floats)
- Error handling (domain errors, undefined operations)





\section{Examples}


\subsection{ Python Data Scientist - SymPy Migration }

Quick comparison of SymPy vs MathHook syntax

\begin{lstlisting}
# SymPy syntax (familiar to data scientists)
# from sympy import symbols, simplify
# x, y = symbols('x y')
# expr = (x + y)**2

# MathHook syntax (similar but faster)
from mathhook import Expression

x = Expression.symbol('x')
y = Expression.symbol('y')
expr = (x.add(y)).pow(2)
simplified = expr.simplify()

\end{lstlisting}




\subsection{ Node.js Developer - Web Form Parsing }

Parse user input LaTeX from web forms

\begin{lstlisting}
# Not applicable for Node.js path

\end{lstlisting}




\subsection{ Rust Programmer - Custom Extension }

Extend Universal Function Registry with custom function

\begin{lstlisting}
# Not applicable for Rust path

\end{lstlisting}




\subsection{ Mathematics Educator - Step-by-Step }

Generate educational explanations for students

\begin{lstlisting}
from mathhook import Expression

x = Expression.symbol('x')
expr = (x.add(1)).mul(x.sub(1))

explanation = expr.explain_simplification()
for step in explanation.steps:
    print(f"{step.title}: {step.description}")

\end{lstlisting}




\subsection{ Computational Scientist - Symbolic Jacobian }

Generate Jacobian matrix for nonlinear system

\begin{lstlisting}
from mathhook import Expression

x = Expression.symbol('x')
y = Expression.symbol('y')

# System of equations
f1 = x.pow(2).add(y)
f2 = x.mul(y)

# Compute Jacobian symbolically
df1_dx = f1.derivative(x)
df1_dy = f1.derivative(y)
df2_dx = f2.derivative(x)
df2_dy = f2.derivative(y)

jacobian = Expression.matrix([
    [df1_dx, df1_dy],
    [df2_dx, df2_dy]
])

\end{lstlisting}







\end{document}
