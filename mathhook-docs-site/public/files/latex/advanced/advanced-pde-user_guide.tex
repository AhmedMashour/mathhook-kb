\documentclass[11pt,a4paper]{article}
\usepackage{amsmath}
\usepackage{amssymb}
\usepackage{listings}
\usepackage{xcolor}
\usepackage{hyperref}

\lstset{
    language=Python,
    basicstyle=\ttfamily\small,
    keywordstyle=\color{blue},
    commentstyle=\color{gray},
    stringstyle=\color{red},
    showstringspaces=false,
    breaklines=true,
    frame=single,
    numbers=left,
    numberstyle=\tiny\color{gray}
}

\title{ PDE User Guide - MathHook CAS }
\author{MathHook CAS}
\date{\today}

\begin{document}

\maketitle

\begin{abstract}
Comprehensive user guide for solving partial differential equations with MathHook. Covers
fundamental PDE concepts, classification systems, the method of characteristics for first-order
PDEs, and practical examples including transport equations, Burgers' equation, and heat equations.
Includes educational features, troubleshooting, and performance considerations.

\end{abstract}


\section{Mathematical Definition}

\begin{equation}
A **partial differential equation (PDE)** is a functional equation involving partial derivatives:

$$F\left(x, y, u, \frac{\partial u}{\partial x}, \frac{\partial u}{\partial y}, \frac{\partial^2 u}{\partial x^2}, \ldots\right) = 0$$

**First-order quasi-linear PDE**:

$$a(x,y,u) \frac{\partial u}{\partial x} + b(x,y,u) \frac{\partial u}{\partial y} = c(x,y,u)$$

**Characteristic equations** (method of characteristics):

$$\frac{dx}{ds} = a(x,y,u), \quad \frac{dy}{ds} = b(x,y,u), \quad \frac{du}{ds} = c(x,y,u)$$

\end{equation}



\section{Introduction}

# PDE User Guide - MathHook CAS

Complete user guide content from the original markdown file, covering:
- Introduction to PDEs and their real-world applications
- PDE classification (by order, type, linearity)
- Getting started with MathHook
- Basic PDE solving examples
- Method of characteristics
- Educational features
- Common patterns and troubleshooting
- Advanced topics and API reference





\section{Examples}


\subsection{ Transport Equation (Step-by-Step) }

Solve transport equation ∂u/∂t + 2·∂u/∂x = 0 with initial condition u(x,0) = sin(x)

\begin{lstlisting}
u = symbol('u')
t = symbol('t')
x = symbol('x')

# Build the PDE
equation = expr(u)
pde = Pde(equation, u, [t, x])

# Solve using method of characteristics
result = method_of_characteristics(pde)

print("Characteristic equations:")
for i, char_eq in enumerate(result.characteristic_equations):
    print(f"  Equation {i + 1}: {char_eq}")

print(f"\nGeneral solution: {result.solution}")

# Apply initial condition: u(x,0) = sin(x)
# Solution: u(x,t) = sin(x - 2t)
solution_with_ic = expr('sin(x + (-2) * t)')
print(f"\nSpecific solution: u(x,t) = {solution_with_ic}")

\end{lstlisting}




\subsection{ Verifying PDE Solution }

Check that solution satisfies the PDE and initial condition

\begin{lstlisting}
from mathhook.derivatives import derivative
from mathhook.simplify import simplify

# Solution: u(x,t) = sin(x - 2*t)
solution = expr('sin(x + (-2) * t)')

# Verify PDE: ∂u/∂t + 2·∂u/∂x = 0
du_dt = derivative(solution, t)
du_dx = derivative(solution, x)

print(f"∂u/∂t = {du_dt}")
# Output: -2*cos(x - 2*t)

print(f"∂u/∂x = {du_dx}")
# Output: cos(x - 2*t)

# Check PDE
lhs = expr(f"{du_dt} + 2 * {du_dx}")
print(f"PDE LHS = {simplify(lhs)}")
# Output: 0 ✓

\end{lstlisting}




\subsection{ Burgers' Equation (Nonlinear) }

Analyze Burgers' equation ∂u/∂t + u·∂u/∂x = 0 showing nonlinear characteristics

\begin{lstlisting}
# Burgers' equation coefficients
u_sym = symbol('u')
coefficients = PdeCoefficients(
    a=expr(1),           # Coefficient of ∂u/∂t
    b=expr(u_sym),       # Coefficient of ∂u/∂x (nonlinear!)
    c=expr(0)            # RHS
)

print("Burgers' equation characteristic system:")
print(f"dt/ds = {coefficients.a}")
print(f"dx/ds = {coefficients.b}")  # Note: depends on u!
print(f"du/ds = {coefficients.c}")

# The solution u = F(x - u*t) is implicit (requires solving for u)
# Warning: Can develop shocks where characteristics intersect

\end{lstlisting}




\subsection{ Educational Step-by-Step Solver }

Get detailed explanations of solution process

\begin{lstlisting}
solver = EducationalPDESolver()

u = symbol('u')
x = symbol('x')
t = symbol('t')

equation = expr('u + x')

# Solve with explanations
result, explanation = solver.solve_pde(equation, u, [x, t])

# Display step-by-step explanation
print("Educational Explanation:")
for i, step in enumerate(explanation.steps):
    print(f"Step {i + 1}: {step.title}")
    print(f"  {step.description}")
    print()

\end{lstlisting}







\end{document}
