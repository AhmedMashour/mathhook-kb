\documentclass[11pt,a4paper]{article}
\usepackage{amsmath}
\usepackage{amssymb}
\usepackage{listings}
\usepackage{xcolor}
\usepackage{hyperref}

\lstset{
    language=Python,
    basicstyle=\ttfamily\small,
    keywordstyle=\color{blue},
    commentstyle=\color{gray},
    stringstyle=\color{red},
    showstringspaces=false,
    breaklines=true,
    frame=single,
    numbers=left,
    numberstyle=\tiny\color{gray}
}

\title{ Custom Parsers and Extensions }
\author{MathHook CAS}
\date{\today}

\begin{document}

\maketitle

\begin{abstract}
Extend MathHook's parser for domain-specific mathematical notation.
Add custom functions, operators, preprocessors, and grammar modifications.

\end{abstract}




\section{Introduction}

# Custom Parsers and Extensions

Extend MathHook's parser for domain-specific mathematical notation.

## Understanding Parser Extension

### What Can You Extend?

MathHook's parser is modular and extensible. You can add:

- **Custom Functions**: Domain-specific functions (chemistry, physics, engineering)
- **Custom Operators**: New infix/prefix/postfix operators
- **Custom Notation**: LaTeX macros, specialized symbols
- **Parser Preprocessors**: Transform input before parsing
- **Lexer Tokens**: New token types for specialized syntax

### When to Extend the Parser

**Use Built-In Features When:**
- Standard mathematical notation suffices
- Functions can be named conventionally
- LaTeX or Wolfram notation covers your needs

**Extend the Parser When:**
- Domain-specific notation is essential (chemistry: `→`, physics: `⊗`)
- Custom operators with special precedence
- Proprietary mathematical notation
- Legacy system compatibility

### Architecture Overview

```
Input String
    ↓
Preprocessor (optional) - Transform syntax before parsing
    ↓
Lexer - Tokenize input (recognizes custom tokens)
    ↓
Parser (LALRPOP) - Build expression tree
    ↓
Post-Processor (optional) - Transform parsed expression
    ↓
Expression
```

## Domain-Specific Examples

### Chemistry Notation

Custom parser for chemical equations:
- `→` for yields
- `⇌` for equilibrium
- Subscripts for molecular formulas

### Quantum Mechanics

Specialized notation:
- `⊗` for tensor product
- `⟨|⟩` for inner product
- `[,]` for commutator
- `{,}` for anticommutator

### Financial Mathematics

Finance-specific functions:
- NPV (Net Present Value)
- IRR (Internal Rate of Return)
- FV/PV (Future/Present Value)
- % operator for percentages

### Control Theory

System notation:
- `*` as convolution
- `ℒ` for Laplace transform
- `//` for feedback connection





\section{Examples}


\subsection{ Adding Custom Functions }

Register domain-specific functions

\begin{lstlisting}
from mathhook.parser import ParserBuilder

parser = (ParserBuilder()
    .add_function("erf", "error_function")
    .add_function("Si", "sine_integral")
    .add_function("Ci", "cosine_integral")
    .build())

expr = parser.parse("erf(x) + Si(x)")
# Parsed as: error_function(x) + sine_integral(x)

\end{lstlisting}




\subsection{ Adding Custom Operators }

Define new infix operators with precedence

\begin{lstlisting}
from mathhook.parser import ParserBuilder, Precedence

parser = (ParserBuilder()
    .add_operator("×", "*")
    .add_operator("⊗", "tensor")
    .add_operator_with_precedence(
        "⊕",
        "direct_sum",
        Precedence.ADDITION
    )
    .build())

expr = parser.parse("A ⊗ B")
# Parsed as: tensor(A, B)

\end{lstlisting}




\subsection{ Preprocessor Transformations }

Transform input before parsing

\begin{lstlisting}
from mathhook.parser import ParserBuilder

def preprocess(input_str):
    return (input_str
        .replace("→", "->")
        .replace("×", "*")
        .replace("÷", "/"))

parser = (ParserBuilder()
    .add_preprocessor(preprocess)
    .build())

expr = parser.parse("x → ∞")
# Preprocessed to: x -> ∞
# Then parsed normally

\end{lstlisting}




\subsection{ Domain-Specific Parser (Chemistry) }

Complete chemistry equation parser

\begin{lstlisting}
from mathhook.parser import ParserBuilder

def create_chemistry_parser():
    return (ParserBuilder()
        .add_operator("→", "yields")
        .add_operator("⇌", "equilibrium")
        .add_operator("+", "plus")
        .add_preprocessor(expand_chemical_formulas)
        .add_postprocessor(balance_equation)
        .build())

parser = create_chemistry_parser()
reaction = parser.parse("H₂ + O₂ → H₂O")
balanced = reaction.balance()  # 2H₂ + O₂ → 2H₂O

\end{lstlisting}




\subsection{ Custom LaTeX Macros }

Expand LaTeX macros before parsing

\begin{lstlisting}
from mathhook.parser import LatexParserBuilder

parser = (LatexParserBuilder()
    .add_macro(r"\RR", r"\mathbb{R}")
    .add_macro(r"\CC", r"\mathbb{C}")
    .add_macro(r"\NN", r"\mathbb{N}")
    .add_macro(r"\dd", r"\mathrm{d}")
    .build())

expr = parser.parse(r"f: \RR \to \CC")
# Expands to: f: \mathbb{R} \to \mathbb{C}

\end{lstlisting}







\end{document}
