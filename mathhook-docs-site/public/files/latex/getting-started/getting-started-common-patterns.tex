\documentclass[11pt,a4paper]{article}
\usepackage{amsmath}
\usepackage{amssymb}
\usepackage{listings}
\usepackage{xcolor}
\usepackage{hyperref}

\lstset{
    language=Python,
    basicstyle=\ttfamily\small,
    keywordstyle=\color{blue},
    commentstyle=\color{gray},
    stringstyle=\color{red},
    showstringspaces=false,
    breaklines=true,
    frame=single,
    numbers=left,
    numberstyle=\tiny\color{gray}
}

\title{ Common Patterns }
\author{MathHook CAS}
\date{\today}

\begin{document}

\maketitle

\begin{abstract}
Common patterns and best practices when using MathHook, including macro usage
guidelines, polynomial construction, substitution patterns, function composition,
matrix operations, error handling, performance patterns, and educational features.
Includes detailed pitfalls to avoid.

\end{abstract}




\section{Introduction}

# Common Patterns

This chapter covers common patterns and best practices when using MathHook.

## Macro Usage Guidelines

**ALWAYS use macros for:**
- Symbol creation: `symbol!(x)` not `Symbol::new("x")`
- Simple expressions: `expr!(x + y)`
- Function calls: `expr!(sin(x))`

**Use explicit API for:**
- Runtime/loop variables (macros see token 'i', not value)
- Programmatic construction with runtime data
- Dynamic polynomial building

## Building Polynomials

**Fixed Degree**: Use macros with `add:` helper
**Dynamic Degree**: Use explicit API with loops

## Substitution Patterns

Single or multiple variable substitution using HashMap.

## Working with Functions

Create with `expr!` macro, `function!` macro, or `Expression::function()` for
runtime function names. Compose functions by nesting.

## Matrix Patterns

Create from vectors, identity matrices, zero matrices. Perform symbolic matrix
operations.

## Error Handling

Handle parsing errors with `match` on `Result`. Handle solver errors by checking
result type.

## Performance Patterns

- Bulk operations: Use iterators and collect
- Caching results: Store in HashMap keyed by string representation
- Reuse expressions (immutable, cheap to clone)

## Educational Patterns

Use step-by-step explanations and derivative explanations for teaching.

## Common Pitfalls

1. **Runtime variables in macros**: Use explicit API for loop variables
2. **Nested macro calls**: Use intermediate variables
3. **Float equality**: Use epsilon comparison, not ==





\section{Examples}


\subsection{ Macro Usage - Correct Patterns }

When to use macros vs explicit API

\begin{lstlisting}
from mathhook import Expression

# Python uses method chaining
x = Expression.symbol('x')
y = Expression.symbol('y')

expr = x.add(y)
expr = Expression.integer(2).mul(x)
expr = x.pow(2)

\end{lstlisting}




\subsection{ Runtime Variables - Explicit API Required }

Why macros don't work with loop variables

\begin{lstlisting}
from mathhook import Expression

# Python doesn't have compile-time macros
# Always use explicit API (which is fine)

x = Expression.symbol('x')
coefficients = [1, 2, 3]
terms = []
for i, coeff in enumerate(coefficients):
    coeff_expr = Expression.integer(coeff)
    power_expr = Expression.integer(i)
    term = coeff_expr.mul(x.pow(power_expr))
    terms.append(term)
polynomial = Expression.add(terms)

\end{lstlisting}




\subsection{ Building Polynomials - Dynamic Degree }

Construct polynomials with runtime coefficients

\begin{lstlisting}
from mathhook import Expression

def build_polynomial(coefficients, x):
    terms = []
    for i, coeff in enumerate(coefficients):
        coeff_expr = Expression.integer(coeff)
        power = Expression.integer(i)
        term = coeff_expr.mul(x.pow(power))
        terms.append(term)
    return Expression.add(terms)

x = Expression.symbol('x')
poly = build_polynomial([1, -5, 6], x)  # x^2 - 5x + 6

\end{lstlisting}




\subsection{ Substitution - Single and Multiple }

Replace symbols with values

\begin{lstlisting}
from mathhook import Expression

x = Expression.symbol('x')
y = Expression.symbol('y')
expr = x.mul(y).add(x).add(y)

# Single substitution
vars = {'x': Expression.integer(3)}
result = expr.substitute(vars)

# Multiple substitutions
vars = {'x': Expression.integer(2), 'y': Expression.integer(3)}
result = expr.substitute(vars)

\end{lstlisting}




\subsection{ Function Composition }

Compose functions by nesting

\begin{lstlisting}
from mathhook import Expression

x = Expression.symbol('x')

# Build step by step
inner = Expression.function('cos', [x])
composed = Expression.function('sin', [inner])

print(f"Composed function: {composed}")

\end{lstlisting}




\subsection{ Performance - Bulk Operations }

Efficient batch processing

\begin{lstlisting}
from mathhook import Expression

x = Expression.symbol('x')

# Simplify many expressions
expressions = [
    x.add(x),
    x.mul(Expression.integer(1)),
    x.pow(2).add(x.pow(2).neg())
]

simplified = [e.simplify() for e in expressions]

\end{lstlisting}




\subsection{ Performance - Caching Results }

Cache frequently computed expressions

\begin{lstlisting}
from mathhook import Expression

x = Expression.symbol('x')
cache = {}

expr = x.pow(2)
key = str(expr)

if key in cache:
    print("Using cached result")
else:
    result = expr.simplify()
    cache[key] = result

\end{lstlisting}




\subsection{ Common Pitfall - Float Equality }

Never use == for approximate values

\begin{lstlisting}
from mathhook import Expression

# WRONG - comparing floats directly
val1 = 3.14
val2 = 3.14000000001
# if val1 == val2:  # BAD!

# CORRECT - use epsilon comparison
tolerance = 1e-10
if abs(val1 - val2) < tolerance:
    print("Values are approximately equal")

# OR use exact rationals
exact = Expression.rational(314, 100)  # Exact 3.14

\end{lstlisting}







\end{document}
