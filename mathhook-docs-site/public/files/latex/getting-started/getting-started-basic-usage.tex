\documentclass[11pt,a4paper]{article}
\usepackage{amsmath}
\usepackage{amssymb}
\usepackage{listings}
\usepackage{xcolor}
\usepackage{hyperref}

\lstset{
    language=Python,
    basicstyle=\ttfamily\small,
    keywordstyle=\color{blue},
    commentstyle=\color{gray},
    stringstyle=\color{red},
    showstringspaces=false,
    breaklines=true,
    frame=single,
    numbers=left,
    numberstyle=\tiny\color{gray}
}

\title{ Basic Usage }
\author{MathHook CAS}
\date{\today}

\begin{document}

\maketitle

\begin{abstract}
Comprehensive guide to using MathHook including expression creation with macros
and constructors, simplification, pattern matching, symbol manipulation, number
types, constants, and function expressions.

\end{abstract}




\section{Introduction}

# Basic Usage

This chapter provides a guide to using MathHook in your projects.

## Expression Creation

### Using Macros

The recommended way to create expressions is using the `expr!` and `symbol!` macros
for clean, readable code.

### Using Constructors

For programmatic construction, use explicit constructors like `Expression::integer()`,
`Expression::rational()`, `Expression::float()`.

## Simplification

Simplification transforms expressions to their canonical form by combining like terms,
applying identities, and evaluating constants.

## Pattern Matching

Work with expression structure using Rust's pattern matching on `Expression` enum
variants: `Add`, `Mul`, `Pow`, etc.

## Working with Symbols

Symbols represent variables in expressions. Symbols with the same name are equal.

## Number Types

MathHook supports integers (exact, arbitrary precision), rationals (exact fractions),
floats (approximate), and complex numbers.

## Constants

Built-in mathematical constants: π, e, i (imaginary unit), φ (golden ratio),
γ (Euler-Mascheroni constant).

## Function Expressions

Create function calls using `expr!` macro or `function!` macro for elementary
functions: sin, cos, tan, log, etc.





\section{Examples}


\subsection{ Expression Creation - Macros }

Create expressions using ergonomic macros

\begin{lstlisting}
from mathhook import Expression

x = Expression.symbol('x')
y = Expression.symbol('y')

# Method chaining for expressions
expr1 = x.add(y)
expr2 = Expression.integer(2).mul(x)
expr3 = x.pow(2)

# Complex expressions
expr4 = x.add(1).mul(x.sub(1))

\end{lstlisting}




\subsection{ Expression Creation - Constructors }

Programmatic construction with explicit API

\begin{lstlisting}
from mathhook import Expression

# Numbers
int_val = Expression.integer(42)
float_val = Expression.float(3.14)
rational_val = Expression.rational(3, 4)  # 3/4

# Operations
sum_val = Expression.integer(1).add(Expression.integer(2))
product_val = Expression.integer(2).mul(x)

\end{lstlisting}




\subsection{ Simplification }

Transform expressions to canonical form

\begin{lstlisting}
from mathhook import Expression

x = Expression.symbol('x')

# Combine like terms
expr = x.add(x)
simplified = expr.simplify()
# Result: 2*x

# Apply identities
expr = x.mul(Expression.integer(1))
simplified = expr.simplify()
# Result: x

# Evaluate constants
expr = Expression.integer(2).add(Expression.integer(3))
simplified = expr.simplify()
# Result: 5

\end{lstlisting}




\subsection{ Pattern Matching (Rust) }

Inspect expression structure with pattern matching

\begin{lstlisting}
# Python doesn't have Rust-style pattern matching
# Use type checking instead
from mathhook import Expression

x = Expression.symbol('x')
y = Expression.symbol('y')
test_expr = x.add(y)

# Check expression type
if test_expr.is_add():
    print("Addition expression")

\end{lstlisting}




\subsection{ Number Types }

Different number representations in MathHook

\begin{lstlisting}
from mathhook import Expression

# Integers (exact)
int_val = Expression.integer(123456789)

# Rationals (exact fractions)
frac = Expression.rational(22, 7)  # 22/7 ≈ π

# Floats (approximate)
float_val = Expression.float(3.14159265359)

# Complex numbers
complex_val = Expression.complex(
    Expression.integer(3),
    Expression.integer(4)
)  # 3 + 4i

\end{lstlisting}




\subsection{ Mathematical Constants }

Built-in mathematical constants

\begin{lstlisting}
from mathhook import Expression

pi = Expression.pi()
e = Expression.e()
i = Expression.i()              # imaginary unit
phi = Expression.golden_ratio()
gamma = Expression.euler_gamma()

\end{lstlisting}




\subsection{ Function Expressions }

Create elementary function calls

\begin{lstlisting}
from mathhook import Expression

x = Expression.symbol('x')

# Elementary functions
sin_x = Expression.function('sin', [x])
cos_x = Expression.function('cos', [x])
log_x = Expression.function('log', [x])

\end{lstlisting}







\end{document}
