\documentclass[11pt,a4paper]{article}
\usepackage{amsmath}
\usepackage{amssymb}
\usepackage{listings}
\usepackage{xcolor}
\usepackage{hyperref}

\lstset{
    language=Python,
    basicstyle=\ttfamily\small,
    keywordstyle=\color{blue},
    commentstyle=\color{gray},
    stringstyle=\color{red},
    showstringspaces=false,
    breaklines=true,
    frame=single,
    numbers=left,
    numberstyle=\tiny\color{gray}
}

\title{ Glossary }
\author{MathHook CAS}
\date{\today}

\begin{document}

\maketitle

\begin{abstract}
Comprehensive glossary of technical terms used throughout MathHook documentation,
covering computer algebra, mathematical concepts, and performance optimization terminology.

\end{abstract}




\section{Introduction}

# Glossary

## A

**AST (Abstract Syntax Tree)**: Tree representation of mathematical expressions.

**Assumption**: Constraint on symbol values (e.g., positive, real, integer).

## C

**CAS (Computer Algebra System)**: Software for symbolic mathematics.

**Canonical Form**: Standard representation of expressions for reliable equality.

**Cache Line**: 64-byte memory chunk that CPUs load at once.

## D

**Domain**: Set of valid input values for a function.

**Derivative**: Rate of change of a function.

## E

**Expression**: Mathematical formula represented as a tree structure.

**Ergonomic API**: User-friendly, intuitive programming interface.

## I

**Immutable**: Cannot be changed after creation.

**Implicit Multiplication**: `2x` interpreted as `2 * x`.

## L

**LALRPOP**: Parser generator tool used by MathHook.

**LaTeX**: Typesetting system for mathematical notation.

## R

**Rational**: Exact fraction representation (numerator/denominator).

## S

**SIMD**: Single Instruction Multiple Data - parallel computation technique.

**Simplification**: Transforming expressions to canonical form.

**Symbol**: Mathematical variable (e.g., x, y, θ).

**String Interning**: Storing one copy of each unique string for fast comparison.

## T

**Thread Safety**: Safe to use from multiple threads simultaneously.

## Z

**Zero-Copy**: Processing without making intermediate copies of data.





\section{Examples}





\end{document}
