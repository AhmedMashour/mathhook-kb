\documentclass[11pt,a4paper]{article}
\usepackage{amsmath}
\usepackage{amssymb}
\usepackage{listings}
\usepackage{xcolor}
\usepackage{hyperref}

\lstset{
    language=Python,
    basicstyle=\ttfamily\small,
    keywordstyle=\color{blue},
    commentstyle=\color{gray},
    stringstyle=\color{red},
    showstringspaces=false,
    breaklines=true,
    frame=single,
    numbers=left,
    numberstyle=\tiny\color{gray}
}

\title{ Polynomial Division and Factorization }
\author{MathHook CAS}
\date{\today}

\begin{document}

\maketitle

\begin{abstract}
Polynomial division algorithms including long division, exact division, and factorization capabilities
such as square-free factorization, resultant, and discriminant computation.

\end{abstract}


\section{Mathematical Definition}

\begin{equation}
**Polynomial Long Division**: For polynomials $f(x), g(x)$ with $\deg(g) \leq \deg(f)$:

$$f(x) = g(x) \cdot q(x) + r(x)$$

where $q(x)$ is the quotient and $r(x)$ is the remainder with $\deg(r) < \deg(g)$.

**Resultant**: The resultant $\text{Res}(f, g)$ of polynomials $f, g$ of degrees $m, n$ is:

$$\text{Res}(f, g) = a_n^m \cdot b_m^n \cdot \prod_{i,j} (\alpha_i - \beta_j)$$

where $\alpha_i, \beta_j$ are roots of $f, g$ respectively. Properties:
- $\text{Res}(f, g) = 0 \iff f$ and $g$ share a common root
- Computed as determinant of Sylvester matrix

**Discriminant**: For polynomial $f(x)$ of degree $n$ with leading coefficient $a_n$:

$$\text{disc}(f) = \frac{(-1)^{n(n-1)/2}}{a_n} \cdot \text{Res}(f, f')$$

Properties:
- $\text{disc}(f) = 0 \iff f$ has a repeated root
- For quadratic $ax^2 + bx + c$: $\text{disc} = b^2 - 4ac$

\end{equation}



\section{Introduction}

This chapter covers polynomial division algorithms and factorization capabilities in MathHook.

## Polynomial Division

### Long Division

The standard polynomial long division algorithm computes quotient and remainder.

### Exact Division

When you expect the division to be exact (zero remainder), use exact division which
returns an error if the division is not exact.

## Factorization

### Common Factor Extraction

Extract the greatest common factor from all terms.

### Numeric Coefficient Factoring

Separate the numeric coefficient from the polynomial.

### Square-Free Factorization

Decompose a polynomial into square-free factors.

## Resultant and Discriminant

### Resultant

The resultant of two polynomials is zero if and only if they share a common root.

### Discriminant

The discriminant indicates whether a polynomial has repeated roots.

### Coprimality Test

Check if two polynomials are coprime (GCD is constant).





\section{Examples}


\subsection{ Polynomial Long Division }

Compute quotient and remainder using standard division algorithm

\begin{lstlisting}
from mathhook import expr, symbol
from mathhook.polynomial.algorithms import polynomial_long_division

x = symbol('x')

# Divide (x^2 - 1) by (x - 1)
dividend = expr('x^2 - 1')
divisor = expr('x - 1')

quotient, remainder = polynomial_long_division(dividend, divisor, x)

# quotient = x + 1
# remainder = 0
# Verify: dividend = divisor * quotient + remainder

\end{lstlisting}




\subsection{ Exact Division }

Division that errors if remainder is non-zero

\begin{lstlisting}
from mathhook import expr, symbol
from mathhook.polynomial.algorithms import exact_division

x = symbol('x')

# x^2 / x = x (exact)
dividend = expr('x^2')
divisor = expr('x')

try:
    quotient = exact_division(dividend, divisor, x)
    print(f"Exact quotient: {quotient}")
except Exception as e:
    print(f"Division not exact: {e}")

\end{lstlisting}




\subsection{ Trait-Based Division }

Use PolynomialArithmetic trait for ergonomic API

\begin{lstlisting}
from mathhook import expr, symbol

x = symbol('x')

f = expr('x^3 - 1')
g = expr('x - 1')

# Returns (quotient, remainder)
q, r = f.poly_div(g, x)
# q = x^2 + x + 1
# r = 0

\end{lstlisting}




\subsection{ Polynomial Resultant }

Test for common roots using resultant

\begin{lstlisting}
from mathhook import expr, symbol
from mathhook.polynomial.algorithms import polynomial_resultant

x = symbol('x')

# p1 = x - 1
p1 = expr('x - 1')
# p2 = x - 2
p2 = expr('x - 2')

res = polynomial_resultant(p1, p2, x)
# Resultant is non-zero (distinct roots)

\end{lstlisting}




\subsection{ Polynomial Discriminant }

Detect repeated roots using discriminant

\begin{lstlisting}
from mathhook import expr, symbol
from mathhook.polynomial.algorithms import polynomial_discriminant

x = symbol('x')

# (x - 1)^2 = x^2 - 2x + 1 (repeated root at 1)
poly = expr('x^2 - 2*x + 1')

disc = polynomial_discriminant(poly, x)
# Discriminant = 0 (repeated root)

\end{lstlisting}







\end{document}
