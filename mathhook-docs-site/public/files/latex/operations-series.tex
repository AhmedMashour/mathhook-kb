\documentclass[11pt,a4paper]{article}
\usepackage{amsmath}
\usepackage{amssymb}
\usepackage{listings}
\usepackage{xcolor}
\usepackage{hyperref}

\lstset{
    language=Python,
    basicstyle=\ttfamily\small,
    keywordstyle=\color{blue},
    commentstyle=\color{gray},
    stringstyle=\color{red},
    showstringspaces=false,
    breaklines=true,
    frame=single,
    numbers=left,
    numberstyle=\tiny\color{gray}
}

\title{ Series Expansions }
\author{MathHook CAS}
\date{\today}

\begin{document}

\maketitle

\begin{abstract}
Expand functions as infinite series for numerical approximation and analysis.

\end{abstract}


\section{Mathematical Definition}

\begin{equation}
**Taylor's Theorem:**
If $f(x)$ is infinitely differentiable at $x = a$, then:
$$f(x) = \sum_{n=0}^{\infty} \frac{f^{(n)}(a)}{n!} (x - a)^n$$

Expanded form:
$$f(x) = f(a) + f'(a)(x-a) + \frac{f''(a)}{2!}(x-a)^2 + \frac{f'''(a)}{3!}(x-a)^3 + \cdots$$

**Maclaurin Series (Special Case: a = 0):**
$$f(x) = \sum_{n=0}^{\infty} \frac{f^{(n)}(0)}{n!} x^n$$

**Common Series:**
- $e^x = 1 + x + \frac{x^2}{2!} + \frac{x^3}{3!} + \cdots$
- $\sin(x) = x - \frac{x^3}{3!} + \frac{x^5}{5!} - \cdots$
- $\cos(x) = 1 - \frac{x^2}{2!} + \frac{x^4}{4!} - \cdots$
- $\ln(1+x) = x - \frac{x^2}{2} + \frac{x^3}{3} - \cdots$ for $|x| < 1$

**Radius of Convergence ($R$):**
The series converges for $|x - a| < R$ and may diverge for $|x - a| > R$.

\end{equation}






\section{Examples}


\subsection{ Maclaurin Series (Expansion at x=0) }

Standard functions at x = 0

\begin{lstlisting}
from mathhook import symbol, series, exp, sin, cos

x = symbol('x')

# exp(x)
exp_series = series(exp(x), x, 0, n=5)
# Result: 1 + x + x^2/2 + x^3/6 + x^4/24 + x^5/120

# sin(x)
sin_series = series(sin(x), x, 0, n=7)
# Result: x - x^3/6 + x^5/120 - x^7/5040

\end{lstlisting}




\subsection{ Taylor Series at Arbitrary Points }

Expand around any point a

\begin{lstlisting}
from mathhook import symbol, series, sin, exp, log, pi

x = symbol('x')

# sin(x) at x = π/2
sin_at_pi_2 = series(sin(x), x, pi/2, n=5)

# exp(x) at x = 1
exp_at_1 = series(exp(x), x, 1, n=5)

# ln(x) at x = 1
log_at_1 = series(log(x), x, 1, n=5)

\end{lstlisting}




\subsection{ Laurent Series (Negative Powers) }

For functions with singularities

\begin{lstlisting}
from mathhook import symbol, laurent_series, exp

x = symbol('x')

# 1/x near x = 0
pole = 1/x
laurent = laurent_series(pole, x, 0, -1, 5)
# Result: x^(-1)

# exp(1/x) at x = 0
exp_pole = exp(1/x)
laurent2 = laurent_series(exp_pole, x, 0, -10, 0)

\end{lstlisting}







\end{document}
