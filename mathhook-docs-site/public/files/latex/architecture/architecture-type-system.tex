\documentclass[11pt,a4paper]{article}
\usepackage{amsmath}
\usepackage{amssymb}
\usepackage{listings}
\usepackage{xcolor}
\usepackage{hyperref}

\lstset{
    language=Python,
    basicstyle=\ttfamily\small,
    keywordstyle=\color{blue},
    commentstyle=\color{gray},
    stringstyle=\color{red},
    showstringspaces=false,
    breaklines=true,
    frame=single,
    numbers=left,
    numberstyle=\tiny\color{gray}
}

\title{ Type System }
\author{MathHook CAS}
\date{\today}

\begin{document}

\maketitle

\begin{abstract}
MathHook's type system design and constraints. Covers Expression, Number, and Symbol types
with their memory layout and performance characteristics.

\end{abstract}




\section{Introduction}

# Type System

This chapter covers internal implementation details of MathHook's type system.

## Core Types

- **Expression**: 32-byte AST node (cache-optimized)
- **Number**: 16-byte tagged union (integer/rational/float/complex)
- **Symbol**: String interning for O(1) equality

## Type Constraints

### Expression Constraint

Expressions are exactly 32 bytes to fit two expressions per 64-byte cache line.
This is a non-negotiable architectural constraint that provides:

- Efficient memory access patterns
- Improved CPU cache utilization
- 10-100x speedup over Python-based systems

### Number Constraint

Numbers are 16 bytes to fit within the Expression type. The tagged union supports:

- Integer (arbitrary precision via pointer to heap)
- Rational (numerator/denominator pair)
- Float (f64)
- Complex (two f64s)

### Symbol Interning

Symbols use string interning for O(1) equality comparisons:

- First occurrence allocates and stores in global table
- Subsequent uses return pointer to existing string
- Equality is pointer comparison (no string comparison needed)

## Type Safety

MathHook's type system provides:

- **Compile-time safety**: Types checked at compile time
- **Mathematical correctness**: Operations preserve mathematical properties
- **Domain enforcement**: Domain restrictions respected (sqrt, log, division by zero)





\section{Examples}





\end{document}
