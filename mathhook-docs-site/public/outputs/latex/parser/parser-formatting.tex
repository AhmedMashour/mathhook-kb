\documentclass[11pt,a4paper]{article}
\usepackage{amsmath}
\usepackage{amssymb}
\usepackage{listings}
\usepackage{xcolor}
\usepackage{hyperref}

\lstset{
    language=Python,
    basicstyle=\ttfamily\small,
    keywordstyle=\color{blue},
    commentstyle=\color{gray},
    stringstyle=\color{red},
    showstringspaces=false,
    breaklines=true,
    frame=single,
    numbers=left,
    numberstyle=\tiny\color{gray}
}

\title{ Expression Formatting }
\author{MathHook CAS}
\date{\today}

\begin{document}

\maketitle

\begin{abstract}
Format mathematical expressions for display in multiple notations.
Supports LaTeX, Unicode, Wolfram, and custom formatters for different output targets.

\end{abstract}




\section{Introduction}

# Expression Formatting

Format mathematical expressions for display in multiple notations.

## Understanding Expression Formatting

### What is Expression Formatting?

Expression formatting converts internal `Expression` structures to human-readable or machine-parseable strings. MathHook provides multiple output formats:

- **LaTeX**: Academic papers, presentations, web rendering (MathJax/KaTeX)
- **Unicode**: Pretty terminal output, notebooks, readable display
- **Wolfram**: Mathematica/Wolfram Language compatibility
- **String**: Rust Debug format for debugging

### How It Works (Architecture)

**Formatter Trait**:
```rust
pub trait Formatter {
    fn format(&self, expr: &Expression) -> String;
}
```

**Implementations**:
- `LaTeXFormatter` - Recursive descent with LaTeX command generation
- `UnicodeFormatter` - Unicode mathematical symbols (superscripts, subscripts)
- `WolframFormatter` - Wolfram Language bracket notation
- Type-aware formatting for noncommutative symbols

## Format Comparison

| **Expression** | **LaTeX** | **Unicode** | **Wolfram** |
|----------------|-----------|-------------|-------------|
| x² + 2x + 1 | `x^{2} + 2 \cdot x + 1` | `x² + 2·x + 1` | `x^2 + 2*x + 1` |
| sin(x) | `\sin(x)` | `sin(x)` | `Sin[x]` |
| √2 | `\sqrt{2}` | `√2` | `Sqrt[2]` |
| π·r² | `\pi \cdot r^{2}` | `π·r²` | `Pi*r^2` |

## Unicode Mathematical Symbols

Supported Unicode ranges:
- **Superscripts**: ⁰¹²³⁴⁵⁶⁷⁸⁹
- **Subscripts**: ₀₁₂₃₄₅₆₇₈₉
- **Greek**: α β γ δ ε ζ η θ ι κ λ μ ν ξ ο π ρ σ τ υ φ χ ψ ω
- **Operators**: · × ÷ ± ∓ ≤ ≥ ≠ ≈ ∞
- **Roots**: √ ∛ ∜
- **Set theory**: ∈ ∉ ⊂ ⊃ ⊆ ⊇ ∪ ∩ ∅





\section{Examples}


\subsection{ Basic Formatting }

Format expressions in different notations

\begin{lstlisting}
from mathhook import symbol, expr
from mathhook.formatter import LatexFormatter, UnicodeFormatter, WolframFormatter

x = symbol('x')
expr_obj = expr('x^2 + 2*x + 1')

# LaTeX
latex = LatexFormatter().format(expr_obj)
print(latex)   # x^{2} + 2 \cdot x + 1

# Unicode (pretty-print)
unicode = UnicodeFormatter().format(expr_obj)
print(unicode) # x² + 2·x + 1

# Wolfram
wolfram = WolframFormatter().format(expr_obj)
print(wolfram) # x^2 + 2*x + 1

\end{lstlisting}




\subsection{ Type-Aware Formatting }

Noncommutative symbols formatted correctly

\begin{lstlisting}
from mathhook import symbol, expr
from mathhook.formatter import LatexFormatter

# Matrix symbols (bold)
A = symbol('A', type='matrix')
B = symbol('B', type='matrix')
matrix_expr = expr('A * B')

formatter = LatexFormatter()
print(formatter.format(matrix_expr))
# Output: \mathbf{A}\mathbf{B}

# Operator symbols (hat)
p = symbol('p', type='operator')
x = symbol('x', type='operator')
op_expr = expr('p * x')

print(formatter.format(op_expr))
# Output: \hat{p}\hat{x}

\end{lstlisting}




\subsection{ Customized LaTeX Output }

Configure formatter behavior

\begin{lstlisting}
from mathhook import symbol, expr
from mathhook.formatter import LatexFormatter

# Configure formatter
formatter = LatexFormatter(
    precision=6,
    explicit_multiplication=True,
    compact_fractions=False
)

expr_obj = expr('2*x / 3')
print(formatter.format(expr_obj))
# Output: \frac{2 \cdot x}{3}

\end{lstlisting}




\subsection{ Educational Step Formatting }

Format step-by-step explanations

\begin{lstlisting}
from mathhook import symbol, expr
from mathhook.formatter import LatexFormatter

x = symbol('x')
expr_obj = expr('x^2 + 2*x + 1')

# Generate step-by-step LaTeX
formatter = LatexFormatter()

print(f"Step 1: Start with {formatter.format(expr_obj)}")
factored = expr_obj.factor()  # (x+1)^2
print(f"Step 2: Factor as {formatter.format(factored)}")

\end{lstlisting}







\end{document}
