\documentclass[11pt,a4paper]{article}
\usepackage{amsmath}
\usepackage{amssymb}
\usepackage{listings}
\usepackage{xcolor}
\usepackage{hyperref}

\lstset{
    language=Python,
    basicstyle=\ttfamily\small,
    keywordstyle=\color{blue},
    commentstyle=\color{gray},
    stringstyle=\color{red},
    showstringspaces=false,
    breaklines=true,
    frame=single,
    numbers=left,
    numberstyle=\tiny\color{gray}
}

\title{ Mathematical Notation }
\author{MathHook CAS}
\date{\today}

\begin{document}

\maketitle

\begin{abstract}
Documentation of mathematical notation used throughout MathHook, including LaTeX support,
standard notation, Wolfram Language syntax, and operator precedence rules.

\end{abstract}




\section{Introduction}

# Mathematical Notation

This appendix documents the mathematical notation used throughout MathHook.

## LaTeX Support

MathHook can parse standard LaTeX mathematical notation:

- `\frac{a}{b}` - Fractions
- `\sqrt{x}` - Square root
- `x^{2}` - Exponentiation
- `\sin(x)`, `\cos(x)`, `\tan(x)` - Trigonometric functions
- `\log(x)`, `\ln(x)` - Logarithms
- `\sum`, `\prod`, `\int` - Summation, product, integral
- Greek letters: `\alpha`, `\beta`, `\gamma`, etc.

## Standard Notation

- `2*x` or `2x` - Multiplication (implicit multiplication supported)
- `x^2` - Exponentiation
- `x/y` - Division
- `sin(x)` - Functions
- `|x|` - Absolute value

## Wolfram Language

MathHook also supports Wolfram Language syntax:

- `Power[x, 2]` - Exponentiation
- `Sin[x]` - Functions
- `D[expr, x]` - Derivatives
- `Integrate[expr, x]` - Integration

## Operator Precedence

1. Function application: `sin(x)`, `log(y)`
2. Exponentiation: `^` (right-associative)
3. Multiplication/Division: `*`, `/` (left-associative)
4. Addition/Subtraction: `+`, `-` (left-associative)





\section{Examples}





\end{document}
