\documentclass[11pt,a4paper]{article}
\usepackage{amsmath}
\usepackage{amssymb}
\usepackage{listings}
\usepackage{xcolor}
\usepackage{hyperref}

\lstset{
    language=Python,
    basicstyle=\ttfamily\small,
    keywordstyle=\color{blue},
    commentstyle=\color{gray},
    stringstyle=\color{red},
    showstringspaces=false,
    breaklines=true,
    frame=single,
    numbers=left,
    numberstyle=\tiny\color{gray}
}

\title{ Separation of Variables for PDEs }
\author{MathHook CAS}
\date{\today}

\begin{document}

\maketitle

\begin{abstract}
Separation of variables is the fundamental technique for solving linear partial differential
equations (PDEs) with boundary conditions. This method transforms a PDE into a system of
ordinary differential equations (ODEs) that can be solved independently, then combines the
solutions into an infinite series.

\end{abstract}


\section{Mathematical Definition}

\begin{equation}
For a PDE with two independent variables ($x$ and $t$), the **product ansatz** assumes:

$$u(x,t) = X(x) \cdot T(t)$$

where $X(x)$ depends **only** on spatial variable $x$ and $T(t)$ depends **only** on
temporal variable $t$.

\end{equation}



\section{Introduction}

# Separation of Variables for PDEs

**Applies to:** Linear second-order PDEs with separable boundary conditions
**Equation types:** Heat equation, wave equation, Laplace equation, and more
**Key idea:** Assume solution is a product of single-variable functions
**MathHook implementation:** Complete workflow from separation to series solution

## Mathematical Background

### What is Separation of Variables?

For a PDE with two independent variables ($x$ and $t$), the **product ansatz** assumes:

$$u(x,t) = X(x) \cdot T(t)$$

where:
- $X(x)$ depends **only** on spatial variable $x$
- $T(t)$ depends **only** on temporal variable $t$

**Key insight:** By substituting this product form into the PDE, we can separate the equation into two independent ODEs—one for $X(x)$ and one for $T(t)$.

### When Does Separation Work?

**Requirements:**

1. **Linear PDE:** The PDE must be linear in $u$ and its derivatives
2. **Separable boundary conditions:** Boundary conditions must only involve one variable
3. **Product domain:** Domain must be a product of intervals (e.g., $[0, L] \times [0, \infty)$)

**Common examples:**
- Heat equation: $\frac{\partial u}{\partial t} = \alpha \frac{\partial^2 u}{\partial x^2}$
- Wave equation: $\frac{\partial^2 u}{\partial t^2} = c^2 \frac{\partial^2 u}{\partial x^2}$
- Laplace equation: $\frac{\partial^2 u}{\partial x^2} + \frac{\partial^2 u}{\partial y^2} = 0$

### The Separation Process (Overview)

1. **Substitute product ansatz** $u(x,t) = X(x)T(t)$ into PDE
2. **Separate variables:** Divide to get $\frac{f(x)}{g(t)} = \text{constant}$
3. **Introduce separation constant** $\lambda$: Each side must equal $-\lambda$
4. **Solve spatial ODE** with boundary conditions → eigenvalues $\lambda_n$ and eigenfunctions $X_n(x)$
5. **Solve temporal ODE** for each $\lambda_n$ → temporal solutions $T_n(t)$
6. **Superposition:** General solution is $u(x,t) = \sum_{n=1}^{\infty} c_n X_n(x) T_n(t)$
7. **Apply initial conditions** → determine coefficients $c_n$ (Fourier series)





\section{Examples}


\subsection{ Heat Equation with Dirichlet BCs }

Solve 1D heat equation with fixed boundary conditions

\begin{lstlisting}
from mathhook import symbol, expr
from mathhook.pde import Pde, BoundaryCondition, InitialCondition, separate_variables

u = symbol('u')
x = symbol('x')
t = symbol('t')

pde = Pde(u, u, [x, t])

# Boundary conditions
bc_left = BoundaryCondition.dirichlet_at(x, expr('0'), expr('0'))
bc_right = BoundaryCondition.dirichlet_at(x, expr('pi'), expr('0'))
bcs = [bc_left, bc_right]

# Initial condition
ic = InitialCondition.value(expr('sin(x)'))
ics = [ic]

solution = separate_variables(pde, bcs, ics)
# Result: eigenvalues [1, 4, 9, 16, ...], eigenfunctions [sin(x), sin(2x), ...]

\end{lstlisting}




\subsection{ Wave Equation }

Solve 1D wave equation with Dirichlet boundary conditions

\begin{lstlisting}
from mathhook import symbol, expr
from mathhook.pde import Pde, BoundaryCondition, InitialCondition, separate_variables

u = symbol('u')
x = symbol('x')
t = symbol('t')
L = symbol('L')

pde = Pde(u, u, [x, t])

bc_left = BoundaryCondition.dirichlet_at(x, expr('0'), expr('0'))
bc_right = BoundaryCondition.dirichlet_at(x, L, expr('0'))
bcs = [bc_left, bc_right]

ic_displacement = InitialCondition.value(expr('sin(pi*x/L)'))
ic_velocity = InitialCondition.derivative(expr('0'))
ics = [ic_displacement, ic_velocity]

solution = separate_variables(pde, bcs, ics)

\end{lstlisting}




\subsection{ Laplace Equation on Rectangle }

Solve Laplace's equation on rectangular domain

\begin{lstlisting}
from mathhook import symbol, expr
from mathhook.pde import Pde, BoundaryCondition, separate_variables

u = symbol('u')
x = symbol('x')
y = symbol('y')
a = symbol('a')

pde = Pde(u, u, [x, y])

bc_left = BoundaryCondition.dirichlet_at(x, expr('0'), expr('0'))
bc_right = BoundaryCondition.dirichlet_at(x, a, expr('0'))
bcs = [bc_left, bc_right]

ics = []  # Laplace is elliptic

solution = separate_variables(pde, bcs, ics)

\end{lstlisting}







\end{document}
