\documentclass[11pt,a4paper]{article}
\usepackage{amsmath}
\usepackage{amssymb}
\usepackage{listings}
\usepackage{xcolor}
\usepackage{hyperref}

\lstset{
    language=Python,
    basicstyle=\ttfamily\small,
    keywordstyle=\color{blue},
    commentstyle=\color{gray},
    stringstyle=\color{red},
    showstringspaces=false,
    breaklines=true,
    frame=single,
    numbers=left,
    numberstyle=\tiny\color{gray}
}

\title{ Limits }
\author{MathHook CAS}
\date{\today}

\begin{document}

\maketitle

\begin{abstract}
Compute limits of expressions as variables approach values, infinity, or points of discontinuity.

\end{abstract}


\section{Mathematical Definition}

\begin{equation}
**Epsilon-Delta Definition (ε-δ):**
$$\lim_{x \to a} f(x) = L$$
means: For every $\varepsilon > 0$, there exists $\delta > 0$ such that:
$$0 < |x - a| < \delta \implies |f(x) - L| < \varepsilon$$

**Limit Laws:**
1. **Sum/Difference:** $\lim_{x \to a} [f(x) \pm g(x)] = \lim_{x \to a} f(x) \pm \lim_{x \to a} g(x)$
2. **Product:** $\lim_{x \to a} [f(x) \cdot g(x)] = \lim_{x \to a} f(x) \cdot \lim_{x \to a} g(x)$
3. **Quotient:** $\lim_{x \to a} \frac{f(x)}{g(x)} = \frac{\lim_{x \to a} f(x)}{\lim_{x \to a} g(x)}$ (if denominator $\neq 0$)

**L'Hôpital's Rule (0/0 or ∞/∞):**
$$\lim_{x \to a} \frac{f(x)}{g(x)} = \lim_{x \to a} \frac{f'(x)}{g'(x)}$$
(if the limit on the right exists)

\end{equation}






\section{Examples}


\subsection{ Direct Substitution }

For continuous functions, substitute directly

\begin{lstlisting}
from mathhook import symbol, limit, pi

x = symbol('x')

# Limit: lim(x→2) x² = 4
expr1 = x**2
limit1 = limit(expr1, x, 2)
# Result: 4

# Limit: lim(x→π) sin(x) = 0
expr2 = sin(x)
limit2 = limit(expr2, x, pi)
# Result: 0

\end{lstlisting}




\subsection{ L'Hôpital's Rule (0/0 Form) }

Use derivatives to resolve indeterminate forms

\begin{lstlisting}
from mathhook import symbol, limit, sin, cos

x = symbol('x')

# Limit: lim(x→0) sin(x)/x = 1
expr = sin(x)/x
result = limit(expr, x, 0)
# Result: 1

# Limit: lim(x→0) (1 - cos(x))/x²
expr2 = (1 - cos(x))/x**2
result2 = limit(expr2, x, 0)
# Result: 1/2

\end{lstlisting}




\subsection{ Limits at Infinity }

Behavior as x approaches ±∞

\begin{lstlisting}
from mathhook import symbol, limit, oo

x = symbol('x')

# Limit: lim(x→∞) (2x² + 1)/(x² + 3)
expr1 = (2*x**2 + 1)/(x**2 + 3)
limit1 = limit(expr1, x, oo)
# Result: 2

# Limit: lim(x→∞) (x + 1)/(x² + 1)
expr2 = (x + 1)/(x**2 + 1)
limit2 = limit(expr2, x, oo)
# Result: 0

\end{lstlisting}







\end{document}
