\documentclass[11pt,a4paper]{article}
\usepackage{amsmath}
\usepackage{amssymb}
\usepackage{listings}
\usepackage{xcolor}
\usepackage{hyperref}

\lstset{
    language=Python,
    basicstyle=\ttfamily\small,
    keywordstyle=\color{blue},
    commentstyle=\color{gray},
    stringstyle=\color{red},
    showstringspaces=false,
    breaklines=true,
    frame=single,
    numbers=left,
    numberstyle=\tiny\color{gray}
}

\title{ LaTeX Parsing and Mathematical Notation }
\author{MathHook CAS}
\date{\today}

\begin{document}

\maketitle

\begin{abstract}
Parse and generate LaTeX notation for mathematical expressions. Full bidirectional
support: LaTeX → Expression and Expression → LaTeX. Automatic type inference for
matrix symbols (\mathbf{A}), operator symbols (\hat{p}), and implicit multiplication.

\end{abstract}




\section{Introduction}

# LaTeX Parsing and Notation

## Overview

MathHook provides comprehensive LaTeX support:
- **Bidirectional**: Parse LaTeX → Expression, Expression → LaTeX
- **Type Inference**: `\mathbf{A}` creates matrix symbols, `\hat{p}` creates operators
- **Implicit Multiplication**: Handles `2x`, `\pi x`, `(a)(b)` correctly
- **150+ Commands**: Functions, symbols, operators, calculus notation

## Architecture

### Two-Stage Processing

**1. Lexer (Token Generation)**:
- Inserts implicit multiplication tokens (`2x` → `2*x`)
- Classifies tokens (number, identifier, function, operator)
- O(1) HashMap lookups for LaTeX commands (`\sin`, `\pi`, `\alpha`)

**2. Parser (LALRPOP Grammar)**:
- LR(1) parser with operator precedence
- Right-associative exponentiation: `2^3^4` → `2^(3^4)`
- Context-aware function resolution

### Performance
- >100K simple expressions/second
- Thread-local caching for common expressions
- Zero-copy string processing where possible

## Supported LaTeX

### Greek Letters
- Lowercase: `\alpha`, `\beta`, `\gamma`, ..., `\omega`
- Uppercase: `\Gamma`, `\Delta`, `\Theta`, ..., `\Omega`

### Mathematical Constants
- `\pi`: π (pi)
- `e`: Euler's number
- `\phi`: Golden ratio
- `i`: Imaginary unit

### Fractions and Roots
- `\frac{a}{b}`: Fractions
- `\sqrt{x}`: Square root
- `\sqrt[n]{x}`: nth root

### Trigonometric Functions
- `\sin`, `\cos`, `\tan`, `\cot`, `\sec`, `\csc`
- `\arcsin`, `\arccos`, `\arctan`
- `\sinh`, `\cosh`, `\tanh`

### Calculus Notation
- `\int`: Integral
- `\frac{d}{dx}`: Derivative
- `\lim`: Limit
- `\sum`: Summation
- `\prod`: Product

### Matrix Notation
- `\mathbf{A}`: Matrix symbol (bold, noncommutative)
- `\hat{p}`: Operator symbol (quantum mechanics)
- `\begin{matrix}...\end{matrix}`: Matrix construction





\section{Examples}


\subsection{ Basic LaTeX Parsing }

Parse standard mathematical expressions

\begin{lstlisting}
from mathhook.parser import parse_latex

# Basic arithmetic
expr = parse_latex(r"2 + 3 \cdot 4")  # 2 + 3*4

# Fractions
expr = parse_latex(r"\frac{x^2 + 1}{x - 1}")

# Functions
expr = parse_latex(r"\sin(x) + \cos(y)")

# Square roots
expr = parse_latex(r"\sqrt{x^2 + y^2}")

# Exponents
expr = parse_latex(r"e^{-x^2}")

\end{lstlisting}




\subsection{ Greek Letters and Constants }

Parse Greek symbols and mathematical constants

\begin{lstlisting}
from mathhook.parser import parse_latex

# Greek symbols
expr = parse_latex(r"\alpha + \beta + \gamma")

# Gamma function
expr = parse_latex(r"\Gamma(n)")

# Constants
expr = parse_latex(r"\pi r^2")
expr = parse_latex(r"e^{i\pi} + 1")
expr = parse_latex(r"\phi = \frac{1+\sqrt{5}}{2}")

\end{lstlisting}




\subsection{ Matrix and Operator Symbols }

Automatic type inference from LaTeX notation

\begin{lstlisting}
from mathhook.parser import parse_latex

# Matrix symbols (automatic inference)
expr = parse_latex(r"\mathbf{A} \mathbf{B}")
# Creates matrix symbols A, B

# Operator symbols
expr = parse_latex(r"\hat{p} \hat{x}")
# Creates operator symbols p, x

# Mixed
expr = parse_latex(r"x \mathbf{A}")

\end{lstlisting}




\subsection{ Generating LaTeX Output }

Convert expressions back to LaTeX

\begin{lstlisting}
from mathhook import symbol
from mathhook.formatter import to_latex

x = symbol('x')

# Simple expression
expr = x**2 / 2
latex = to_latex(expr)
# Returns: "\frac{x^{2}}{2}"

# Matrix expression
A = symbol('A', matrix=True)
B = symbol('B', matrix=True)
expr = A * B
latex = to_latex(expr)
# Returns: "\mathbf{A}\mathbf{B}"

\end{lstlisting}




\subsection{ Implicit Multiplication }

Automatic insertion of multiplication operators

\begin{lstlisting}
from mathhook.parser import parse_latex

# Implicit multiplication handled automatically
expr = parse_latex("2x")           # 2*x
expr = parse_latex("(a)(b)")       # a*b
expr = parse_latex(r"\sin(x)\cos(y)")  # sin(x)*cos(y)
expr = parse_latex(r"2\pi r")      # 2*π*r

\end{lstlisting}




\subsection{ Calculus Notation }

Parse derivatives, integrals, limits

\begin{lstlisting}
from mathhook.parser import parse_latex

# Derivative
expr = parse_latex(r"\frac{d}{dx} x^2")

# Integral
expr = parse_latex(r"\int x^2 \, dx")

# Definite integral
expr = parse_latex(r"\int_0^1 x^2 \, dx")

# Limit
expr = parse_latex(r"\lim_{x \to 0} \frac{\sin(x)}{x}")

# Summation
expr = parse_latex(r"\sum_{i=1}^{n} i^2")

\end{lstlisting}







\end{document}
