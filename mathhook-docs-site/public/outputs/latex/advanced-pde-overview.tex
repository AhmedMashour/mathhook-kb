\documentclass[11pt,a4paper]{article}
\usepackage{amsmath}
\usepackage{amssymb}
\usepackage{listings}
\usepackage{xcolor}
\usepackage{hyperref}

\lstset{
    language=Python,
    basicstyle=\ttfamily\small,
    keywordstyle=\color{blue},
    commentstyle=\color{gray},
    stringstyle=\color{red},
    showstringspaces=false,
    breaklines=true,
    frame=single,
    numbers=left,
    numberstyle=\tiny\color{gray}
}

\title{ Partial Differential Equations (PDEs) }
\author{MathHook CAS}
\date{\today}

\begin{document}

\maketitle

\begin{abstract}
Comprehensive overview of partial differential equations in MathHook CAS.
Covers mathematical foundations, classification, solution methods, and current capabilities.

\end{abstract}


\section{Mathematical Definition}

\begin{equation}
A second-order linear PDE in two independent variables has the general form:

$$A \frac{\partial^2 u}{\partial x^2} + B \frac{\partial^2 u}{\partial x \partial y} + C \frac{\partial^2 u}{\partial y^2} + D \frac{\partial u}{\partial x} + E \frac{\partial u}{\partial y} + Fu = G$$

where:
- $u(x,y)$ is the unknown function
- $A, B, C, D, E, F, G$ are coefficients (may depend on $x$, $y$, or $u$)
- $x, y$ are independent variables (typically spatial coordinates or time)

\end{equation}



\section{Introduction}

# Partial Differential Equations (PDEs)

## What Are PDEs?

Partial Differential Equations (PDEs) describe relationships involving functions of multiple variables and their partial derivatives. Unlike Ordinary Differential Equations (ODEs) which involve functions of a single variable, PDEs govern phenomena that vary in space **and** time.

### Why PDEs Matter

PDEs are the mathematical language of:
1. **Physics**: Heat conduction, wave propagation, electromagnetic fields, quantum mechanics
2. **Engineering**: Structural analysis, fluid dynamics, signal processing, control systems
3. **Finance**: Option pricing (Black-Scholes), risk modeling
4. **Biology**: Population dynamics, pattern formation, reaction-diffusion systems
5. **Computer Graphics**: Image processing, surface modeling, fluid simulation

## MathHook PDE Module Capabilities

### What MathHook Provides (Version 7.5/10)

✅ **Core Functionality**:
- PDE classification via discriminant ($B^2 - 4AC$)
- Heat equation solver (1D, Dirichlet boundary conditions)
- Wave equation solver (1D, Dirichlet boundary conditions)
- Laplace equation solver (2D rectangular domains, Dirichlet boundary conditions)
- Eigenvalue computation for standard boundary conditions
- Registry-based solver dispatch (O(1) lookup)
- Symbolic solution representation

✅ **Mathematical Correctness**:
- Verified against SymPy reference implementation
- Correct eigenvalue formulas
- Proper separation of variables structure
- Accurate boundary condition handling

### Current Limitations (Honestly Documented)

⚠️ **Symbolic Fourier Coefficients**:
- Solutions contain symbolic coefficients ($A_1, A_2, A_3, \ldots$)
- Numerical evaluation requires symbolic integration (Phase 2)
- Example: Heat equation returns $u(x,t) = \sum A_n \sin(\lambda_n x) e^{-\lambda_n \alpha t}$ with $A_n$ symbolic

⚠️ **Limited Boundary Conditions**:
- Only Dirichlet (fixed value) boundary conditions fully supported
- Neumann (derivative) and Robin (mixed) BCs planned for Phase 2

⚠️ **Standard Equations Only**:
- Supports heat, wave, and Laplace equations
- General nonlinear PDEs not yet supported

## Solution Methodology: Separation of Variables

All MathHook PDE solvers use the **separation of variables** technique.





\section{Examples}


\subsection{ Registry-Based Solver Dispatch }

Automatic PDE classification and solver selection using O(1) registry lookup

\begin{lstlisting}
from mathhook import symbol, expr, Pde, PDESolverRegistry

# Create registry
registry = PDESolverRegistry()

# Define PDE
u = symbol('u')
x = symbol('x')
t = symbol('t')
equation = expr(x + t)  # Heat equation pattern
pde = Pde(equation, u, [x, t])

# Automatic solving
solution = registry.solve(pde)

print(f"Solution: {solution.solution}")
print(f"Eigenvalues: {solution.get_eigenvalues()}")

\end{lstlisting}







\end{document}
