\documentclass[11pt,a4paper]{article}
\usepackage{amsmath}
\usepackage{amssymb}
\usepackage{listings}
\usepackage{xcolor}
\usepackage{hyperref}

\lstset{
    language=Python,
    basicstyle=\ttfamily\small,
    keywordstyle=\color{blue},
    commentstyle=\color{gray},
    stringstyle=\color{red},
    showstringspaces=false,
    breaklines=true,
    frame=single,
    numbers=left,
    numberstyle=\tiny\color{gray}
}

\title{ Complex Number Operations }
\author{MathHook CAS}
\date{\today}

\begin{document}

\maketitle

\begin{abstract}
Work with complex numbers in MathHook including imaginary unit i, complex
arithmetic, polar form, Euler's formula, and operations like conjugate,
magnitude, and argument.

\end{abstract}


\section{Mathematical Definition}

\begin{equation}
Complex number: $$z = a + bi$$ where $a, b \in \mathbb{R}$ and $i^2 = -1$

Polar form: $$z = r e^{i\theta} = r(\cos\theta + i\sin\theta)$$

where $r = |z| = \sqrt{a^2 + b^2}$ and $\theta = \arg(z) = \arctan(b/a)$

\end{equation}



\section{Introduction}

# Complex Number Operations

MathHook provides comprehensive support for complex number arithmetic,
conversions between rectangular and polar forms, and complex functions.

## Creating Complex Numbers

```rust
use mathhook::Expression;

// Imaginary unit
let i = Expression::i();

// Complex number: 3 + 4i
let z = expr!(3 + 4*i);

// Pure imaginary: 5i
let w = expr!(5*i);
```

## Operations

### Addition/Subtraction
Component-wise: (a + bi) ± (c + di) = (a ± c) + (b ± d)i

### Multiplication
(a + bi)(c + di) = (ac - bd) + (ad + bc)i

### Division
Division by conjugate multiplication

### Conjugate
conj(a + bi) = a - bi

### Magnitude
|a + bi| = √(a² + b²)

### Argument
arg(a + bi) = arctan(b/a)





\section{Examples}


\subsection{ Basic Complex Arithmetic }



\begin{lstlisting}
i = expr('I')
z1 = expr('3 + 4*I')
z2 = expr('1 - 2*I')

sum_z = z1 + z2       # 4 + 2*I
product = z1 * z2     # 11 - 2*I

\end{lstlisting}




\subsection{ Euler's Formula }



\begin{lstlisting}
theta = symbol('theta')
euler = exp(I * theta)

# Expands to: cos(theta) + I*sin(theta)
expanded = expand(euler)

\end{lstlisting}




\subsection{ Polar Form Conversion }



\begin{lstlisting}
z = expr('3 + 4*I')

magnitude = abs(z)  # 5
angle = arg(z)      # atan(4/3)

# Polar form
polar = magnitude * exp(I * angle)

\end{lstlisting}







\end{document}
