\documentclass[11pt,a4paper]{article}
\usepackage{amsmath}
\usepackage{amssymb}
\usepackage{listings}
\usepackage{xcolor}
\usepackage{hyperref}

\lstset{
    language=Python,
    basicstyle=\ttfamily\small,
    keywordstyle=\color{blue},
    commentstyle=\color{gray},
    stringstyle=\color{red},
    showstringspaces=false,
    breaklines=true,
    frame=single,
    numbers=left,
    numberstyle=\tiny\color{gray}
}

\title{ GCD Algorithms }
\author{MathHook CAS}
\date{\today}

\begin{document}

\maketitle

\begin{abstract}
Multiple GCD (Greatest Common Divisor) algorithms for polynomials, optimized for different use cases
including univariate, multivariate, and modular GCD using Zippel's algorithm.

\end{abstract}


\section{Mathematical Definition}

\begin{equation}
For polynomials $f, g \in R[x]$ over a ring $R$, the greatest common divisor $\gcd(f, g)$ is the
monic polynomial $d$ of maximum degree such that:

$$d \mid f \quad \text{and} \quad d \mid g$$

and for any other polynomial $h$ where $h \mid f$ and $h \mid g$, we have $h \mid d$.

**Euclidean Algorithm**: For univariate polynomials over a field:

$$\gcd(f, g) = \begin{cases}
f & \text{if } g = 0 \\
\gcd(g, f \bmod g) & \text{otherwise}
\end{cases}$$

**Zippel's Modular Algorithm**:
1. Extract content: $f = c_f \cdot f_p$, $g = c_g \cdot g_p$
2. Compute $\gcd(f_p, g_p)$ in $\mathbb{Z}_p[x]$ for prime $p$
3. Use CRT to reconstruct $\gcd$ in $\mathbb{Z}[x]$

\end{equation}



\section{Introduction}

MathHook provides multiple GCD (Greatest Common Divisor) algorithms for polynomials, optimized for different use cases.

## Algorithm Selection Guide

### Quick Decision Tree

```
Need GCD of two polynomials?
├─ Both are i64 integers? → integer_gcd(a, b)
├─ Don't know the structure? → polynomial_gcd(&p1, &p2)
├─ Single variable (x)? → univariate_gcd(&p1, &p2, &x)
├─ Need cofactors too? → modular_gcd_univariate(&p1, &p2, &x)
└─ Multiple variables (x, y, z)? → multivariate_gcd(&p1, &p2, &[x, y, z])
```

## Zippel's Modular GCD Algorithm

For performance-critical applications, the Zippel algorithm provides industrial-strength GCD computation using modular arithmetic.

### How It Works

1. **Content Extraction**: Separate integer content from primitive parts
2. **Prime Selection**: Choose primes that don't divide leading coefficients
3. **Modular GCD**: Compute GCD in Z_p[x] using Euclidean algorithm
4. **CRT Reconstruction**: Combine results from multiple primes using Chinese Remainder Theorem
5. **Trial Division**: Verify the result divides both inputs

### Configuration Options

- **max_eval_points**: Maximum number of evaluation points per variable
- **use_sparse**: Whether to use sparse optimization
- **starting_prime_idx**: Prime index to start with

## Performance Characteristics

| Algorithm | Complexity | Best For |
|-----------|------------|----------|
| Integer GCD | O(log(min(a,b))) | Small integers |
| Univariate Modular | O(d^2) | Single variable polynomials |
| Multivariate Zippel | O(d^n) | Sparse multivariate |
| Groebner-based | Doubly exponential | Ideal membership |

Where `d` is degree and `n` is number of variables.





\section{Examples}


\subsection{ General-Purpose GCD }

Use PolynomialGcdOps trait for automatic algorithm selection

\begin{lstlisting}
from mathhook import expr, symbol
from mathhook.polynomial import PolynomialOps

x = symbol('x')

# f = x^2 - 1 = (x-1)(x+1)
f = expr('x^2 - 1')
# g = x^2 - 2x + 1 = (x-1)^2
g = expr('x^2 - 2*x + 1')

# Compute GCD
gcd = f.polynomial_gcd(g)
# gcd = x - 1

# Compute LCM
lcm = f.polynomial_lcm(g)
# lcm = (x-1)^2(x+1)

\end{lstlisting}




\subsection{ Univariate Modular GCD with Cofactors }

Returns GCD and cofactors for Bezout identity verification

\begin{lstlisting}
from mathhook import expr, symbol
from mathhook.polynomial.algorithms import modular_gcd_univariate

x = symbol('x')
f = expr('x^2 - 1')
g = expr('x - 1')

# Returns (gcd, cofactor_f, cofactor_g)
gcd, cof_f, cof_g = modular_gcd_univariate(f, g, x)

# Verify: f = gcd * cof_f, g = gcd * cof_g

\end{lstlisting}




\subsection{ Multivariate GCD with Zippel Algorithm }

Compute GCD for polynomials in multiple variables

\begin{lstlisting}
from mathhook import expr, symbol
from mathhook.polynomial.algorithms import multivariate_gcd_zippel, MultivariateGcdConfig

x = symbol('x')
y = symbol('y')

# f = x*y, g = x*y + x
f = expr('x*y')
g = expr('x*y + x')

config = MultivariateGcdConfig()
gcd, _, _ = multivariate_gcd_zippel(f, g, [x, y], config)
# gcd = x

\end{lstlisting}




\subsection{ Content and Primitive Part Decomposition }

Fundamental operation for GCD computation

\begin{lstlisting}
from mathhook.polynomial.algorithms import extract_content, primitive_part

coeffs = [6, 12, 18]  # 6 + 12x + 18x^2

# Extract content (GCD of coefficients)
content = extract_content(coeffs)  # 6

# Get primitive part
cont, pp = primitive_part(coeffs)  # (6, [1, 2, 3])

\end{lstlisting}







\end{document}
