\documentclass[11pt,a4paper]{article}
\usepackage{amsmath}
\usepackage{amssymb}
\usepackage{listings}
\usepackage{xcolor}
\usepackage{hyperref}

\lstset{
    language=Python,
    basicstyle=\ttfamily\small,
    keywordstyle=\color{blue},
    commentstyle=\color{gray},
    stringstyle=\color{red},
    showstringspaces=false,
    breaklines=true,
    frame=single,
    numbers=left,
    numberstyle=\tiny\color{gray}
}

\title{ Educational Message Registry }
\author{MathHook CAS}
\date{\today}

\begin{document}

\maketitle

\begin{abstract}
The message registry system provides organized, mappable, hashable educational content separated from code logic. Instead of hardcoding explanatory text throughout the codebase, MathHook maintains a centralized registry of educational messages that can be customized, internationalized, and adapted for different audiences.

\end{abstract}




\section{Introduction}

# Educational Message Registry

> 📍 **Navigation:** [Step-by-Step](./step-by-step.md) | [Educational API](./api.md) | [Previous: Getting Started](../getting-started/learning-paths.md)

The message registry system provides organized, mappable, hashable educational content separated from code logic. Instead of hardcoding explanatory text throughout the codebase, MathHook maintains a centralized registry of educational messages that can be customized, internationalized, and adapted for different audiences.

## What is the Message Registry?

**Problem:** Hardcoding educational text in code makes it difficult to:
- Customize explanations for different student levels
- Translate content to other languages
- Maintain consistent educational messaging
- A/B test different explanations
- Update content without code changes

**Solution:** The message registry provides a centralized, indexed system where:
- Messages are stored separately from code logic
- Each message has a unique hash for fast lookup
- Templates support dynamic substitution
- Content can be customized per audience without touching code

**Learning Journey:** After understanding [step-by-step explanations](./step-by-step.md), learn how to customize the language. Then explore [programmatic API integration](./api.md).

## Architecture

### Core Components

```rust
use mathhook::educational::message_registry::{
    MessageCategory,
    MessageType,
    MessageKey,
    MessageHashSystem,
    MessageBuilder,
};
```

### Message Key Structure

Every message is uniquely identified by a composite key:

```rust
pub struct MessageKey {
    pub category: String,        // Domain: "linear_equation", "calculus", etc.
    pub message_type: String,    // Type: "introduction", "strategy", "result"
    pub variant: u32,            // Alternative phrasing (0, 1, 2, ...)
    pub hash: u64,               // Fast lookup hash
    pub template_params: Vec<String>,  // Required substitutions
}
```

**Hash System for Performance:**

Messages are hashed for O(1) lookup:

$$
\text{hash} = \text{fnv1a}(\text{category} \oplus \text{type} \oplus \text{variant})
$$

This allows instant message retrieval without string comparison.

## Message Categories

### Algebra Messages

Messages for algebraic operations including linear equations, quadratic equations, polynomial equations, and general algebraic simplification.

### Calculus Messages

Messages for calculus operations including derivatives (power rule, chain rule), integration (by parts, substitution), and limits.

### Solver Messages

Messages for equation solving including system equations (substitution, elimination), matrix methods, and solution verification.

### ODE Messages

Messages for ordinary differential equations including separable equations, linear first-order equations, and exact equations.

**Separable ODE Form:**

$$
\frac{dy}{dx} = g(x) \cdot h(y) \implies \frac{dy}{h(y)} = g(x) \, dx
$$

### PDE Messages

Messages for partial differential equations including heat equation, wave equation, and Laplace equation.

**Heat Equation:**

$$
\frac{\partial u}{\partial t} = \alpha \frac{\partial^2 u}{\partial x^2}
$$

### Noncommutative Algebra Messages

Messages for matrix and operator algebra where order matters. Critical property: For matrices, $AB \neq BA$ in general, so left and right division are different.





\section{Examples}


\subsection{ Basic Message Usage }

Create and use educational messages with template substitution

\begin{lstlisting}
intro = MessageBuilder(
    MessageCategory.LINEAR_EQUATION,
    MessageType.INTRODUCTION,
    0  # variant
) \
.with_substitution("equation", "2x + 3 = 7") \
.with_substitution("variable", "x") \
.build()

print(intro.description)
# Output: "We have a linear equation in the form ax + b = c. To solve for x, we'll isolate the variable."

\end{lstlisting}




\subsection{ Calculus Message with Power Rule }

Generate educational message for derivative explanation

\begin{lstlisting}
derivative_msg = MessageBuilder(
    MessageCategory.CALCULUS,
    MessageType.DERIVATIVE_POWER_RULE,
    0
) \
.with_substitution("expression", "x^3") \
.with_substitution("exponent", "3") \
.with_substitution("result", "3x^2") \
.build()

print(derivative_msg.description)
# Output: "Apply the power rule: d/dx(x^3) = 3·x^(3-1) = 3x^2"

\end{lstlisting}




\subsection{ Multiple Variants for Different Audiences }

Use different message variants for beginner, intermediate, and advanced students

\begin{lstlisting}
# Variant 0: Formal mathematical language
formal = MessageBuilder(
    MessageCategory.LINEAR_EQUATION,
    MessageType.STRATEGY,
    0  # variant 0
).build()

# Variant 1: Conversational tone
casual = MessageBuilder(
    MessageCategory.LINEAR_EQUATION,
    MessageType.STRATEGY,
    1  # variant 1
).build()

# Variant 2: Step-by-step procedural
procedural = MessageBuilder(
    MessageCategory.LINEAR_EQUATION,
    MessageType.STRATEGY,
    2  # variant 2
).build()

\end{lstlisting}




\subsection{ Generating Educational Step Sequences }

Generate complete explanation sequences using message registry

\begin{lstlisting}
# Generate complete explanation sequence
steps = EducationalMessageGenerator.linear_equation_steps(
    "2x + 3 = 7",  # equation
    "x",           # variable
    "2"            # solution
)

for step in steps:
    print(step.description)

\end{lstlisting}







\end{document}
