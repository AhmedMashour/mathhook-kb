\documentclass[11pt,a4paper]{article}
\usepackage{amsmath}
\usepackage{amssymb}
\usepackage{listings}
\usepackage{xcolor}
\usepackage{hyperref}

\lstset{
    language=Python,
    basicstyle=\ttfamily\small,
    keywordstyle=\color{blue},
    commentstyle=\color{gray},
    stringstyle=\color{red},
    showstringspaces=false,
    breaklines=true,
    frame=single,
    numbers=left,
    numberstyle=\tiny\color{gray}
}

\title{ Method of Characteristics }
\author{MathHook CAS}
\date{\today}

\begin{document}

\maketitle

\begin{abstract}
The Method of Characteristics is the primary technique for solving first-order
partial differential equations (PDEs). It transforms the PDE into a system of
ordinary differential equations (ODEs) that can be solved along special curves
called characteristic curves.

\end{abstract}


\section{Mathematical Definition}

\begin{equation}
**Quasi-linear PDE:**
$$a(x,y,u) \frac{\partial u}{\partial x} + b(x,y,u) \frac{\partial u}{\partial y} = c(x,y,u)$$

**Characteristic equations:**
$$\frac{dx}{ds} = a(x,y,u), \quad \frac{dy}{ds} = b(x,y,u), \quad \frac{du}{ds} = c(x,y,u)$$

where $s$ is a parameter along the characteristic curve.

**Transport Equation:**
$$\frac{\partial u}{\partial t} + c \cdot \frac{\partial u}{\partial x} = 0$$

**General solution:** $u(x,t) = f(x - ct)$ where $f$ is determined by initial conditions.

**Burgers' Equation (Nonlinear):**
$$\frac{\partial u}{\partial t} + u \frac{\partial u}{\partial x} = 0$$

**Implicit solution:** $u(x,t) = f(x - u(x,t) \cdot t)$

\end{equation}



\section{Introduction}

# Method of Characteristics

The **Method of Characteristics** is the primary technique for solving first-order partial differential equations (PDEs). It transforms the PDE into a system of ordinary differential equations (ODEs) that can be solved along special curves called **characteristic curves**.

## Quick Overview

**Applies to:** First-order quasi-linear PDEs
**Equation form:** $$a(x,y,u) \frac{\partial u}{\partial x} + b(x,y,u) \frac{\partial u}{\partial y} = c(x,y,u)$$
**Key idea:** PDE becomes ODE along characteristic curves
**MathHook implementation:** `method_of_characteristics()` function

## Mathematical Foundation

### Geometric Interpretation

A first-order PDE defines a **direction field** in the $(x,y,u)$ space. The solution surface $u(x,y)$ must be tangent to this direction field everywhere.

**Characteristic curves** are integral curves of this direction field. Along each characteristic, the PDE reduces to an ODE that can be solved.

### The Characteristic System

For the quasi-linear PDE:
$$a(x,y,u) \frac{\partial u}{\partial x} + b(x,y,u) \frac{\partial u}{\partial y} = c(x,y,u)$$

The **characteristic equations** are:
$$\frac{dx}{ds} = a(x,y,u), \quad \frac{dy}{ds} = b(x,y,u), \quad \frac{du}{ds} = c(x,y,u)$$

where $s$ is a parameter along the characteristic curve.

### Solution Strategy

1. **Solve the characteristic system** of ODEs
2. **Obtain parametric solution** $(x(s), y(s), u(s))$
3. **Eliminate parameter $s$** to get implicit solution $u = u(x,y)$
4. **Apply initial/boundary conditions** to determine integration constants

## Complete Examples

### Example 1: Transport Equation

**PDE:** $\frac{\partial u}{\partial t} + c \cdot \frac{\partial u}{\partial x} = 0$

**Physical meaning:** Wave traveling at constant speed $c$

**Characteristic equations:**
$$\frac{dt}{ds} = 1, \quad \frac{dx}{ds} = c, \quad \frac{du}{ds} = 0$$

**Solution:**
- From first two: $x - ct = \text{constant}$
- From third: $u = \text{constant}$ along characteristics
- **General solution:** $u(x,t) = f(x - ct)$ where $f$ is arbitrary

**Initial condition:** If $u(x,0) = g(x)$, then $u(x,t) = g(x - ct)$

### Example 2: Burgers' Equation (Nonlinear)

**PDE:** $\frac{\partial u}{\partial t} + u \frac{\partial u}{\partial x} = 0$

**Physical meaning:** Nonlinear wave with speed depending on amplitude

**Characteristic equations:**
$$\frac{dt}{ds} = 1, \quad \frac{dx}{ds} = u, \quad \frac{du}{ds} = 0$$

**Key insight:** $u$ is constant along each characteristic, but different characteristics have different speeds!

**Solution process:**
- From $\frac{du}{ds} = 0$: $u = u_0$ (constant along characteristic)
- From $\frac{dx}{ds} = u_0$: $x = u_0 t + x_0$
- **Implicit solution:** $u(x,t) = u_0$ where $x_0 = x - u_0 t$

**Initial condition:** $u(x,0) = f(x)$ gives $u(x,t) = f(x - u(x,t) \cdot t)$ (implicit!)

**Shock formation:** Characteristics can intersect, leading to discontinuities (shocks)

### Example 3: General Linear Case

**PDE:** $\frac{\partial u}{\partial x} + 2\frac{\partial u}{\partial y} = 3u$

**Characteristic equations:**
$$\frac{dx}{ds} = 1, \quad \frac{dy}{ds} = 2, \quad \frac{du}{ds} = 3u$$

**Solution:**
- From first two: $y = 2x + C_1$ (characteristic curves are straight lines)
- From third: $u = C_2 e^{3s}$
- Eliminating $s$: $u = C_2 e^{3x}$
- **General solution:** $u(x,y) = g(y - 2x) e^{3x}$ where $g$ is arbitrary

## When to Use Method of Characteristics

**✅ Use when:**
- PDE is first-order
- Need exact analytical solution
- Understanding wave propagation
- Educational demonstrations

**❌ Don't use when:**
- PDE is second-order (use Separation of Variables)
- Need numerical approximation (use finite differences)
- Complex nonlinear PDEs (may require specialized methods)

## Common Pitfalls

### 1. Implicit Solutions
Some PDEs yield **implicit solutions** that cannot be solved for $u$ explicitly.

**Example:** Burgers' equation gives $u(x,t) = f(x - u(x,t) \cdot t)$

**What to do:** Accept implicit form or use numerical methods

### 2. Shock Formation
**Characteristics can intersect** in nonlinear PDEs, causing **discontinuities (shocks)**.

**Example:** Burgers' equation with $u(x,0) = -x$ develops shock at $t = 1$

**What to do:** Use weak solutions or shock-capturing numerics

### 3. Parameter Elimination
Eliminating parameter $s$ can be non-trivial for complex characteristic systems.

**Strategy:** Look for first integrals or invariant combinations





\section{Examples}


\subsection{ Transport Equation Solution }

Solving the transport equation using method of characteristics

\begin{lstlisting}
u = symbol('u')
t = symbol('t')
x = symbol('x')

# Transport equation PDE structure
equation = expr(u)
pde = Pde.new(equation, u, [t, x])

# Solve
solution = method_of_characteristics(pde)
print("Solution: u(x,t) = F(x - ct)")

# With initial condition u(x,0) = sin(x):
print("Specific solution: u(x,t) = sin(x - ct)")

\end{lstlisting}




\subsection{ General Usage Pattern }

Standard pattern for using method of characteristics in MathHook

\begin{lstlisting}
# Define PDE
u = symbol('u')
x = symbol('x')
t = symbol('t')

equation = # build PDE expression
pde = Pde.new(equation, u, [x, t])

# Solve
try:
    solution = method_of_characteristics(pde)
    print(f"Solution: {solution.solution}")
    # Apply initial conditions as needed
except Exception as e:
    print(f"Error: {e}")

\end{lstlisting}







\end{document}
