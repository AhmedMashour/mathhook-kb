\documentclass[11pt,a4paper]{article}
\usepackage{amsmath}
\usepackage{amssymb}
\usepackage{listings}
\usepackage{xcolor}
\usepackage{hyperref}

\lstset{
    language=Python,
    basicstyle=\ttfamily\small,
    keywordstyle=\color{blue},
    commentstyle=\color{gray},
    stringstyle=\color{red},
    showstringspaces=false,
    breaklines=true,
    frame=single,
    numbers=left,
    numberstyle=\tiny\color{gray}
}

\title{ Function Evaluation }
\author{MathHook CAS}
\date{\today}

\begin{document}

\maketitle

\begin{abstract}
MathHook provides a unified, intelligent function evaluation system that handles both symbolic
and numerical computation. The system uses the Universal Function Registry architecture to
dispatch function calls to specialized implementations while maintaining mathematical correctness.

\end{abstract}




\section{Introduction}

# Function Evaluation

MathHook provides a unified, intelligent function evaluation system that handles both symbolic and numerical computation. The system uses the **Universal Function Registry** architecture to dispatch function calls to specialized implementations while maintaining mathematical correctness.

## Overview

Function evaluation in MathHook supports:

- **Elementary functions**: sin, cos, tan, exp, log, sqrt, abs, and their inverses
- **Hyperbolic functions**: sinh, cosh, tanh, and their inverses
- **Special functions**: gamma, zeta, bessel functions
- **Number theory functions**: factorial, binomial coefficients
- **Symbolic evaluation**: Returns exact symbolic results when possible
- **Numerical evaluation**: High-performance numerical approximations
- **Special value recognition**: Automatically simplifies known exact values

## Evaluation Architecture

### Function Intelligence System

Every function in MathHook has associated **intelligence properties** that define:

1. **Domain and Range**: Where the function is defined and what values it can produce
2. **Special Values**: Known exact values (e.g., sin(0) = 0, gamma(1) = 1)
3. **Evaluation Strategy**: How to compute the function symbolically and numerically
4. **Mathematical Properties**: Symmetry, periodicity, derivative rules, etc.

### Evaluation Flow

```
User Expression
      ↓
Function Name + Arguments
      ↓
Universal Registry Lookup
      ↓
Function Properties Dispatch
      ↓
┌─────────────────┬──────────────────┐
│ Special Value?  │ Symbolic Input?  │ Numerical Input?
│ → Exact Result  │ → Keep Symbolic  │ → Numerical Eval
└─────────────────┴──────────────────┘
```

## Performance Characteristics

The function evaluation system is designed for high performance:

- **Registry lookup**: O(1) constant time using hash maps
- **Special value detection**: <50ns for known values
- **Numerical evaluation**: <100ns for elementary functions
- **Total dispatch overhead**: <10ns
- **Bulk evaluation**: SIMD-optimized for arrays of values

## Mathematical Correctness Guarantees

MathHook's function evaluation system provides:

1. **Exact symbolic computation**: Special values return exact results (not floating-point approximations)
2. **Domain checking**: Functions respect their mathematical domains (e.g., log requires positive inputs)
3. **SymPy validation**: All implementations validated against SymPy reference
4. **Numerical stability**: Algorithms chosen for numerical accuracy





\section{Examples}


\subsection{ Elementary Functions }

Evaluating basic trigonometric and exponential functions

\begin{lstlisting}
from mathhook import symbol, expr

x = symbol('x')

sin_x = expr('sin(x)')
cos_x = expr('cos(x)')
exp_x = expr('exp(x)')
log_x = expr('log(x)')

\end{lstlisting}




\subsection{ Special Value Evaluation }

Automatic simplification of known exact values

\begin{lstlisting}
from mathhook import expr

# Trigonometric special values
sin_0 = expr('sin(0)')
assert sin_0.simplify() == expr('0')

cos_0 = expr('cos(0)')
assert cos_0.simplify() == expr('1')

# Exponential and logarithmic
exp_0 = expr('exp(0)')
assert exp_0.simplify() == expr('1')

log_1 = expr('log(1)')
assert log_1.simplify() == expr('0')

\end{lstlisting}




\subsection{ Composite Expression Evaluation }

Mixed symbolic and numeric evaluation

\begin{lstlisting}
from mathhook import symbol, expr

x = symbol('x')
y = symbol('y')

# sqrt(4) evaluates to 2, symbolic parts preserved
composite = expr('sin(x^2 + 1) * cos(y) - sqrt(4)')
result = composite.simplify()
# Result: sin(x^2 + 1) * cos(y) - 2

\end{lstlisting}




\subsection{ Function Composition }

Nested and composed functions

\begin{lstlisting}
from mathhook import symbol, expr

x = symbol('x')

# sin(cos(x))
nested = expr('sin(cos(x))')

# exp(log(x)) simplifies to x
exp_log = expr('exp(log(x))')
simplified = exp_log.simplify()
# Result: x

\end{lstlisting}




\subsection{ Bulk Evaluation }

Efficient numerical evaluation over multiple points

\begin{lstlisting}
from mathhook.functions import FunctionEvaluator

evaluator = FunctionEvaluator()
points = [0.0, 0.5, 1.0, 1.5, 2.0]

# SIMD-optimized evaluation
results = evaluator.evaluate_bulk('sin', points)
print(f"Results: {results}")

\end{lstlisting}







\end{document}
