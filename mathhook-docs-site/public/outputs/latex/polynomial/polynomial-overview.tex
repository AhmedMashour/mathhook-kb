\documentclass[11pt,a4paper]{article}
\usepackage{amsmath}
\usepackage{amssymb}
\usepackage{listings}
\usepackage{xcolor}
\usepackage{hyperref}

\lstset{
    language=Python,
    basicstyle=\ttfamily\small,
    keywordstyle=\color{blue},
    commentstyle=\color{gray},
    stringstyle=\color{red},
    showstringspaces=false,
    breaklines=true,
    frame=single,
    numbers=left,
    numberstyle=\tiny\color{gray}
}

\title{ Polynomial Module Overview }
\author{MathHook CAS}
\date{\today}

\begin{document}

\maketitle

\begin{abstract}
Comprehensive symbolic polynomial manipulation capabilities in MathHook. Implements a trait-based
architecture for automatic classification, property computation, arithmetic operations, and GCD algorithms.

\end{abstract}


\section{Mathematical Definition}

\begin{equation}
A polynomial in variable $x$ over a ring $R$ is an expression of the form:

$$f(x) = a_n x^n + a_{n-1} x^{n-1} + \cdots + a_1 x + a_0$$

where $a_i \in R$ are coefficients and $n \in \mathbb{N}$ is the degree.

For multivariate polynomials in variables $x_1, x_2, \ldots, x_k$:

$$f(x_1, \ldots, x_k) = \sum_{\alpha \in \mathbb{N}^k} c_\alpha x_1^{\alpha_1} \cdots x_k^{\alpha_k}$$

\end{equation}



\section{Introduction}

The polynomial module provides comprehensive symbolic polynomial manipulation capabilities in MathHook. It implements a trait-based architecture for automatic classification, property computation, arithmetic operations, and GCD algorithms.

## Architecture

The module uses **decomposed traits** for clean separation of concerns:

| Trait | Purpose |
|-------|---------|
| `PolynomialClassification` | Type detection and variable extraction |
| `PolynomialProperties` | Degree, leading coefficient, content, primitive part |
| `PolynomialArithmetic` | Division, multiplication, addition |
| `PolynomialGcdOps` | GCD, LCM, cofactors |
| `PolynomialEducational` | Step-by-step explanations (opt-in) |

### Design Philosophy

1. **Automatic Classification**: Users don't need to manually wrap expressions - the system detects polynomial structure automatically
2. **Trait Composition**: Functionality is split into focused traits rather than one monolithic interface
3. **Performance First**: Thread-local LRU caching for expensive operations like degree computation
4. **Educational Support**: Optional step-by-step explanations for learning

## Key Concepts

### Univariate vs Multivariate

| Type | Description | Example |
|------|-------------|---------|
| **Univariate** | Single variable | `x^2 + 2x + 1` |
| **Multivariate** | Multiple variables | `x^2 + xy + y^2` |

The module automatically detects and routes to appropriate algorithms.

### Content and Primitive Part

For a polynomial $f(x) = a_n x^n + \cdots + a_1 x + a_0$:

- **Content**: $\gcd(a_n, \ldots, a_1, a_0)$ - the GCD of all coefficients
- **Primitive Part**: $f(x) / \text{content}(f)$ - the polynomial with content factored out

### Caching Strategy

Property computations are cached using thread-local LRU caching:

- **Automatic**: No user intervention required
- **Thread-safe**: Each thread has its own cache
- **Size-limited**: LRU eviction prevents memory bloat
- **Hash-based**: Uses pointer address + discriminant for fast lookup





\section{Examples}


\subsection{ Basic Polynomial Usage }

Create polynomials and compute properties using trait-based API

\begin{lstlisting}
from mathhook import expr, symbol
from mathhook.polynomial import PolynomialOps

x = symbol('x')

# Create polynomials
f = expr('x^2 + 2*x + 1')
g = expr('x^2 - 1')

# Properties
assert f.degree(x) == 2
assert f.is_polynomial_in([x])

# GCD computation
gcd = f.polynomial_gcd(g)
# gcd = x + 1

\end{lstlisting}




\subsection{ Polynomial Classification }

Automatic detection of polynomial structure and variable extraction

\begin{lstlisting}
from mathhook import expr, symbol

x = symbol('x')
y = symbol('y')

# Automatic detection
poly = expr('x^2 + y*x + 1')
assert poly.is_polynomial()
assert poly.is_polynomial_in([x, y])

# Variable extraction
vars = poly.polynomial_variables()
# vars contains x and y

\end{lstlisting}




\subsection{ Content and Primitive Part }

Extract GCD of coefficients and primitive polynomial

\begin{lstlisting}
from mathhook import expr, symbol

x = symbol('x')
poly = expr('6*x^2 + 9*x + 3')

content = poly.content()         # 3
primitive = poly.primitive_part() # 2*x^2 + 3*x + 1

\end{lstlisting}







\end{document}
