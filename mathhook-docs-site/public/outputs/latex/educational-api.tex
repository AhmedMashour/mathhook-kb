\documentclass[11pt,a4paper]{article}
\usepackage{amsmath}
\usepackage{amssymb}
\usepackage{listings}
\usepackage{xcolor}
\usepackage{hyperref}

\lstset{
    language=Python,
    basicstyle=\ttfamily\small,
    keywordstyle=\color{blue},
    commentstyle=\color{gray},
    stringstyle=\color{red},
    showstringspaces=false,
    breaklines=true,
    frame=single,
    numbers=left,
    numberstyle=\tiny\color{gray}
}

\title{ Educational API }
\author{MathHook CAS}
\date{\today}

\begin{document}

\maketitle

\begin{abstract}
The Educational API provides programmatic access to MathHook's educational features for external applications. Integrate step-by-step explanations, assessment tools, and adaptive learning systems into Learning Management Systems (LMS), mobile apps, and educational platforms.

\end{abstract}




\section{Introduction}

# Educational API

> 📍 **Navigation:** [Step-by-Step](./step-by-step.md) | [Message Registry](./messages.md) | [Advanced Features](../advanced/complex-numbers.md)

The Educational API provides programmatic access to MathHook's educational features for external applications. Integrate step-by-step explanations, assessment tools, and adaptive learning systems into Learning Management Systems (LMS), mobile apps, and educational platforms.

## What is the Educational API?

**Learning Journey:** This is the advanced topic after mastering [step-by-step explanations](./step-by-step.md) and [message customization](./messages.md). Here you'll learn programmatic integration for external applications.

**Purpose:** Enable external applications to:
- Generate educational content programmatically
- Export structured data for machine consumption
- Integrate with Learning Management Systems
- Build adaptive learning applications
- Create assessment and verification tools
- Track student progress systematically

**Design Philosophy:** Dual-format output—human-readable explanations for students AND machine-consumable data for applications.

## API Architecture

### Core Components

```rust
use mathhook::educational::{
    traits::{EducationalOperation, OperationContext},
    enhanced_steps::{EnhancedStep, EnhancedStepExplanation, EnhancedStepBuilder},
    step_by_step::{StepByStepExplanation, Step},
};
```

### Layer Architecture

```
┌─────────────────────────────────────────────┐
│         External Application                 │
│    (LMS, Mobile App, Web Frontend)          │
└─────────────────┬───────────────────────────┘
                  │
                  │ JSON/REST API
                  │
┌─────────────────▼───────────────────────────┐
│      Educational API Layer                   │
│  (EnhancedStep, EducationalOperation)       │
└─────────────────┬───────────────────────────┘
                  │
                  │ Internal API
                  │
┌─────────────────▼───────────────────────────┐
│    Message Registry + Step Generation       │
│  (Templates, Substitution, SmartStepFactory)│
└─────────────────┬───────────────────────────┘
                  │
                  │ Core Operations
                  │
┌─────────────────▼───────────────────────────┐
│      Mathematical Engine                     │
│  (Solving, Simplification, Differentiation) │
└─────────────────────────────────────────────┘
```

## Dual-Format Output

### EnhancedStep Structure

The core dual-format type:

```rust
pub struct EnhancedStep {
    pub step_id: String,
    pub title: String,
    pub human_message: String,      // For students
    pub api_data: StepApiData,      // For machines
    pub message_key: MessageKey,    // For customization
    pub math_context: MathContext,  // Mathematical state
    pub presentation: PresentationHints,
}
```

### Machine-Readable Format

JSON Structure for external applications with metadata, summary, and detailed step information.





\section{Examples}


\subsection{ Basic Dual-Format Output }

Generate both human and machine-readable educational content

\begin{lstlisting}
x = symbol('x')
equation = expr_parse('2*x + 3')

# Get enhanced explanation
explanation = EnhancedStepExplanation(steps)

# Display for students
for step in explanation.steps:
    print(step.human_message)

# Export structured data
json_data = explanation.to_json()
# Parse in external application
data = json.loads(json_data)

\end{lstlisting}




\subsection{ SmartStepFactory Context-Aware Generation }

Generate educational steps based on operation context and difficulty

\begin{lstlisting}
# Generate introduction step for linear equation
intro_step = EnhancedStepBuilder("step_1") \
    .with_human_message(
        "Identify Equation Type",
        "We have a linear equation in one variable"
    ) \
    .with_api_data("linear_equation", "identification", "classify") \
    .with_input("equation", "2x + 3 = 0") \
    .with_input("variable", "x") \
    .with_output("equation_type", "linear") \
    .with_output("degree", "1") \
    .with_math_context("2x + 3 = 0", "x", 0.25) \
    .with_message_key("linear_equation", "introduction", 0) \
    .build()

\end{lstlisting}




\subsection{ Educational Operation Trait Implementation }

Add educational capabilities to mathematical operations

\begin{lstlisting}
class LinearEquationSolver(EducationalOperation):
    def __init__(self, equation, variable):
        self.equation = equation
        self.variable = variable

    def execute_with_steps(self):
        steps = []

        # Step 1: Identify equation type
        steps.append(Step(
            "Identify Equation Type",
            "This is a linear equation ax + b = 0"
        ))

        # Step 2: Isolate variable term
        steps.append(Step(
            "Isolate Variable Term",
            "Subtract constant from both sides"
        ))

        # Step 3: Solve for variable
        steps.append(Step(
            "Solve for Variable",
            "Divide both sides by coefficient"
        ))

        # Perform actual solving
        solution = self.solve_internal()

        explanation = StepByStepExplanation(steps)
        return (solution, explanation)

    def educational_context(self):
        return OperationContext.equation_solving(3)

    def execute_fast(self):
        return self.solve_internal()

    def estimated_steps(self):
        return 3

\end{lstlisting}




\subsection{ LMS Integration with Progress Tracking }

Export educational content to Learning Management System

\begin{lstlisting}
# Generate explanation
explanation = EnhancedStepExplanation(steps)

# Export to JSON
json_data = explanation.to_json()

# Send to LMS via REST API
response = requests.post(
    "https://lms.example.com/api/lessons",
    json=json.loads(json_data)
)

# Track which steps student has viewed
for step in explanation.steps:
    lms_api.mark_step_viewed(
        student_id,
        lesson_id,
        step.step_id
    )

\end{lstlisting}




\subsection{ Assessment and Verification }

Verify student answers and provide feedback

\begin{lstlisting}
def verify_answer(student_answer, expected_solution, variable):
    student_expr = parse_latex(student_answer)

    # Substitute student's answer into original equation
    substituted = original_equation.substitute(variable, student_expr)
    simplified = substituted.simplify()

    return VerificationResult(
        correct=(simplified == Expression.integer(0)),
        student_expression=student_expr,
        substituted_form=substituted,
        explanation=generate_verification_explanation(simplified)
    )

\end{lstlisting}







\end{document}
