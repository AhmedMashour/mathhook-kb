\documentclass[11pt,a4paper]{article}
\usepackage{amsmath}
\usepackage{amssymb}
\usepackage{listings}
\usepackage{xcolor}
\usepackage{hyperref}

\lstset{
    language=Python,
    basicstyle=\ttfamily\small,
    keywordstyle=\color{blue},
    commentstyle=\color{gray},
    stringstyle=\color{red},
    showstringspaces=false,
    breaklines=true,
    frame=single,
    numbers=left,
    numberstyle=\tiny\color{gray}
}

\title{ Quick Start }
\author{MathHook CAS}
\date{\today}

\begin{document}

\maketitle

\begin{abstract}
Get up and running with MathHook in 5 minutes. Learn basic expression creation,
parsing, differentiation, and common operations across Rust, Python, and Node.js.

\end{abstract}




\section{Introduction}

# Quick Start

This guide will get you up and running with MathHook in 5 minutes.

## Your First Expression

Create symbolic expressions using the high-level API with ergonomic macros (Rust)
or method chaining (Python/Node.js).

## Common Operations

- **Parsing LaTeX**: Transform LaTeX notation into symbolic expressions
- **Computing Derivatives**: Automatic symbolic differentiation
- **Solving Equations**: Symbolic equation solving
- **Matrix Operations**: Symbolic linear algebra
- **Step-by-Step Explanations**: Educational features for learning

## Expression Operators

The `expr!` macro (Rust) supports mathematical operators:
- Comparison: `==`, `<`, `>`, `<=`, `>=`
- Method calls: `.abs()`, `.sqrt()`, `.simplify()`
- Power operations: `^`, `**`, `.pow()`

## Common Patterns

- **Creating Expressions Programmatically**: Use macros for compile-time values,
  explicit API for runtime values
- **Substitution**: Replace symbols with values using HashMap
- **Formatting Output**: Standard, LaTeX, and Wolfram notation

## Common Mistakes

- **Runtime Variables in Macros**: Use explicit API for loop variables
- **Precedence Without Parentheses**: Always use explicit grouping
- **Floating Point Comparison**: Use epsilon comparison for numerical values





\section{Examples}


\subsection{ First Expression - Quadratic }

Build and simplify x^2 + 2x + 1

\begin{lstlisting}
from mathhook import Expression

x = Expression.symbol('x')
expr = x.pow(2).add(x.multiply(2)).add(1)
simplified = expr.simplify()

print(f"Original: {expr}")
print(f"Simplified: {simplified}")

\end{lstlisting}




\subsection{ Parsing LaTeX }

Parse LaTeX notation into symbolic expression

\begin{lstlisting}
from mathhook import Parser, ParserConfig

parser = Parser(ParserConfig.default())
expr = parser.parse(r"\frac{x^2 + 1}{2}")
print(expr)

\end{lstlisting}




\subsection{ Computing Derivatives }

Compute first and second derivatives of x^3

\begin{lstlisting}
from mathhook import Expression

x = Expression.symbol('x')
expr = x.pow(3)

derivative = expr.derivative(x)
second_derivative = expr.nth_derivative(x, 2)

print(f"f(x) = {expr}")
print(f"f'(x) = {derivative}")
print(f"f''(x) = {second_derivative}")

\end{lstlisting}




\subsection{ Solving Equations }

Solve x^2 = 4 symbolically

\begin{lstlisting}
from mathhook import Expression, MathSolver

x = Expression.symbol('x')
solver = MathSolver()
equation = Expression.equation(x.pow(2), Expression.integer(4))
solutions = solver.solve(equation, x)

print(f"Solutions: {solutions}")

\end{lstlisting}




\subsection{ Substitution }

Substitute x = 3 into x^2 + 2x + 1

\begin{lstlisting}
from mathhook import Expression

x = Expression.symbol('x')
expr = x.pow(2).add(x.multiply(2)).add(1)

vars = {'x': Expression.integer(3)}
result = expr.substitute(vars)
print(f"Result: {result}")

\end{lstlisting}




\subsection{ Creating Expressions Programmatically }

Use macros for compile-time values, explicit API for runtime

\begin{lstlisting}
from mathhook import Expression

x = Expression.symbol('x')

# Direct creation
expr = x.pow(2).add(3)

# Runtime creation
terms = []
for i in range(5):
    terms.append(
        Expression.mul([
            Expression.integer(i),
            x.pow(Expression.integer(i))
        ])
    )
polynomial = Expression.add(terms)

\end{lstlisting}







\end{document}
