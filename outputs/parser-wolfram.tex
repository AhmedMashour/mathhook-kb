\documentclass[11pt,a4paper]{article}
\usepackage{amsmath}
\usepackage{amssymb}
\usepackage{listings}
\usepackage{xcolor}
\usepackage{hyperref}

\lstset{
    language=Python,
    basicstyle=\ttfamily\small,
    keywordstyle=\color{blue},
    commentstyle=\color{gray},
    stringstyle=\color{red},
    showstringspaces=false,
    breaklines=true,
    frame=single,
    numbers=left,
    numberstyle=\tiny\color{gray}
}

\title{ Wolfram Language Notation }
\author{MathHook CAS}
\date{\today}

\begin{document}

\maketitle

\begin{abstract}
Parse and generate Mathematica/Wolfram Language syntax for compatibility with Wolfram products.
Supports bidirectional conversion between MathHook expressions and Wolfram notation.

\end{abstract}




\section{Introduction}

# Wolfram Language Notation

Parse and generate Mathematica/Wolfram Language syntax for compatibility with Wolfram products.

## Understanding Wolfram Notation

### What is Wolfram Language?

Wolfram Language (used in Mathematica, Wolfram Alpha, Wolfram Cloud) is a symbolic computation language with:

- **Capital letter functions**: `Sin`, `Cos`, `Exp` (not `sin`, `cos`, `exp`)
- **Bracket notation**: Function calls use `[]` (e.g., `Sin[x]`, not `Sin(x)`)
- **Symbolic core**: Everything is an expression tree (similar to MathHook)
- **Pattern matching**: Powerful transformation rules (MathHook: simplification)

### Why Wolfram Compatibility?

- **Academic Migration**: Many researchers use Mathematica
- **Cross-Platform**: Export MathHook results to Wolfram Alpha
- **Data Exchange**: Import/export equations between systems
- **Validation**: Compare MathHook results with Mathematica

### How It Works (Architecture)

**Parser:**
- Recognizes Wolfram function tokens (`Sin`, `Cos`, `D`, `Integrate`)
- Bracket parsing: `f[x, y, z]` → Function call
- PascalCase → snake_case conversion for custom functions

**Formatter:**
- Lowercase → Capitalized function names (`sin` → `Sin`)
- Parentheses → Brackets (`(...)` → `[...]`)
- Operator precedence matching Wolfram

## Wolfram ↔ MathHook Translation Table

| **Operation** | **Wolfram** | **MathHook** | **Notes** |
|---------------|-------------|--------------|-----------|
| Addition | `a + b` | `a + b` | Same |
| Multiplication | `a * b` or `a b` | `a * b` | Same |
| Division | `a / b` | `a / b` | Same |
| Power | `a^b` | `a^b` | Same |
| Function call | `f[x]` | `f(x)` | Brackets vs parens |
| Derivative | `D[f, x]` | `f.derivative(&x, 1)` | Functional vs method |
| Integral | `Integrate[f, x]` | `f.integrate(&x)` | Functional vs method |
| Sqrt | `Sqrt[x]` | `sqrt(x)` | Capital vs lowercase |
| Sin | `Sin[x]` | `sin(x)` | Capital vs lowercase |
| Cos | `Cos[x]` | `cos(x)` | Capital vs lowercase |
| Exp | `Exp[x]` | `exp(x)` | Capital vs lowercase |
| Log | `Log[x]` | `log(x)` | Capital vs lowercase |





\section{Examples}


\subsection{ Basic Wolfram Parsing }

Parse common Wolfram Language expressions

\begin{lstlisting}
from mathhook.parser import parse_wolfram

# Functions (capital letters, brackets)
expr = parse_wolfram("Sin[x]")
expr = parse_wolfram("Exp[x]")
expr = parse_wolfram("Log[x]")

# Powers use ^ (not brackets)
expr = parse_wolfram("x^2")

\end{lstlisting}




\subsection{ Calculus Operations }

Wolfram calculus notation

\begin{lstlisting}
from mathhook.parser import parse_wolfram

# Derivative: D[expr, var]
expr = parse_wolfram("D[x^2, x]")  # 2x

# Integral: Integrate[expr, var]
expr = parse_wolfram("Integrate[x^2, x]")  # x^3/3

# Definite integral: Integrate[expr, {var, a, b}]
expr = parse_wolfram("Integrate[x^2, {x, 0, 1}]")

# Limit: Limit[expr, var -> value]
expr = parse_wolfram("Limit[Sin[x]/x, x -> 0]")

\end{lstlisting}




\subsection{ Generating Wolfram Output }

Format MathHook expressions as Wolfram Language

\begin{lstlisting}
from mathhook import symbol, expr, WolframFormatter

formatter = WolframFormatter()
x = symbol('x')

# Functions
expr_obj = expr('sin(x)')
print(formatter.format(expr_obj))  # Sin[x]

# Complex expressions
expr_obj = expr('(x + 1) / (x - 1)')
print(formatter.format(expr_obj))  # (x + 1)/(x - 1)

\end{lstlisting}




\subsection{ Cross-Platform Validation }

Export to Wolfram for verification

\begin{lstlisting}
from mathhook import symbol, expr, WolframFormatter

# Compute derivative in MathHook
x = symbol('x')
f = expr('x^3 + 2*x^2 + x')
df = f.derivative(x, 1)

# Export to Wolfram for verification
formatter = WolframFormatter()
wolfram_code = formatter.format(df)

print("Verify in Wolfram Alpha:")
print(f"Simplify[{wolfram_code}]")

\end{lstlisting}







\end{document}
