\documentclass[11pt,a4paper]{article}
\usepackage{amsmath}
\usepackage{amssymb}
\usepackage{listings}
\usepackage{xcolor}
\usepackage{hyperref}

\lstset{
    language=Python,
    basicstyle=\ttfamily\small,
    keywordstyle=\color{blue},
    commentstyle=\color{gray},
    stringstyle=\color{red},
    showstringspaces=false,
    breaklines=true,
    frame=single,
    numbers=left,
    numberstyle=\tiny\color{gray}
}

\title{ Function Intelligence System }
\author{MathHook CAS}
\date{\today}

\begin{document}

\maketitle

\begin{abstract}
MathHook's function intelligence registry that enables automatic simplification,
differentiation, and symbolic manipulation of mathematical functions.

\end{abstract}




\section{Introduction}

# Function Intelligence System

This chapter covers internal implementation details of the function intelligence system.

## Function Registry

MathHook maintains a global registry of mathematical functions with their properties:

- **Differentiation rules**: How to differentiate each function
- **Simplification patterns**: Known identities and simplifications
- **Domain restrictions**: Valid input ranges
- **Special values**: Function behavior at key points

## Automatic Simplification

The function intelligence system enables automatic simplification based on:

- Known identities (sin²(x) + cos²(x) = 1)
- Special values (sin(0) = 0, log(1) = 0)
- Composition rules (sin(asin(x)) = x)

## Symbolic Differentiation

Functions in the registry include differentiation rules:

- Elementary functions (sin, cos, exp, log)
- Hyperbolic functions (sinh, cosh, tanh)
- Special functions (gamma, beta, erf)





\section{Examples}





\end{document}
