\documentclass[11pt,a4paper]{article}
\usepackage{amsmath}
\usepackage{amssymb}
\usepackage{listings}
\usepackage{xcolor}
\usepackage{hyperref}

\lstset{
    language=Python,
    basicstyle=\ttfamily\small,
    keywordstyle=\color{blue},
    commentstyle=\color{gray},
    stringstyle=\color{red},
    showstringspaces=false,
    breaklines=true,
    frame=single,
    numbers=left,
    numberstyle=\tiny\color{gray}
}

\title{ Separable Differential Equations }
\author{MathHook CAS}
\date{\today}

\begin{document}

\maketitle

\begin{abstract}
Solve first-order ordinary differential equations that can be separated into functions of x and y independently
\end{abstract}


\section{Mathematical Definition}

\begin{equation}
\frac{dy}{dx} = g(x)h(y) \quad \Rightarrow \quad \int \frac{dy}{h(y)} = \int g(x)dx + C
\end{equation}



\section{Introduction}

Imagine you're a detective investigating a crime scene, and you find a body.
The coroner needs to estimate time of death. How? By measuring body temperature
and using Newton's Law of Cooling—a separable differential equation. From modeling
population growth to predicting how fast your coffee cools, separable ODEs are
everywhere in nature and engineering.




\section{ What Makes an ODE Separable? }

A first-order ODE is **separable** if it can be written in the form:

$$\\frac{dy}{dx} = g(x)h(y)$$

Notice that the right-hand side factors into a function of $x$ multiplied by
a function of $y$. This special structure allows us to algebraically manipulate
the equation to get all $y$ terms on one side and all $x$ terms on the other.

### Recognition Patterns

**Separable:**
- $dy/dx = xy$ (factors as $x \\cdot y$)
- $dy/dx = e^{x+y} = e^x \\cdot e^y$ (exponential property)
- $dy/dx = \\sin(x)\\cos(y)$ (product of functions)

**NOT Separable:**
- $dy/dx = x + y$ (sum, not product)
- $dy/dx = xy + y^2$ (can't factor into g(x)h(y))
- $dy/dx = \\sin(xy)$ (mixed argument)

### The Key Insight

If we can write $dy/dx = g(x)h(y)$, then we can rearrange:

$$\\frac{dy}{h(y)} = g(x)dx$$

Now both sides are ready to integrate independently!


\section{ The Separation of Variables Method }

The solution process has four clear steps:

### Step 1: Verify Separability

Check if the equation can be factored as $dy/dx = g(x)h(y)$.

### Step 2: Separate Variables

Rearrange to get all $y$ terms (including $dy$) on one side and all $x$ terms
(including $dx$) on the other:

$$\\frac{dy}{h(y)} = g(x)dx$$

### Step 3: Integrate Both Sides

$$\\int \\frac{dy}{h(y)} = \\int g(x)dx + C$$

The constant of integration $C$ appears on the right side (convention).

### Step 4: Solve for y (if possible)

Try to solve algebraically for $y$ as an explicit function of $x$. Sometimes
this isn't possible, and we get an **implicit solution** instead (which is
still valid!).

### Example: dy/dx = xy

1. **Verify:** $g(x) = x$, $h(y) = y$ ✓
2. **Separate:** $dy/y = x\\,dx$
3. **Integrate:** $\\ln|y| = x^2/2 + C$
4. **Solve:** $y = Ae^{x^2/2}$ (where $A = \\pm e^C$)


\section{ Initial Value Problems }

An **initial condition** $y(x_0) = y_0$ allows us to find the particular
solution (a specific curve) rather than the general solution (a family of curves).

### Method

1. Find the general solution with constant $C$
2. Substitute the initial condition: $y(x_0) = y_0$
3. Solve for $C$
4. Write the particular solution

### Example: Exponential Growth with Initial Population

**Problem:** $dy/dx = 0.5y$, $y(0) = 100$

**General solution:**
- Separate: $dy/y = 0.5\\,dx$
- Integrate: $\\ln|y| = 0.5x + C$
- Solve: $y = Ae^{0.5x}$

**Apply initial condition:**
- $y(0) = 100 \\Rightarrow Ae^0 = A = 100$

**Particular solution:** $y = 100e^{0.5x}$

This represents a population that doubles every $\\ln(2)/0.5 \\approx 1.386$ time units.


\section{ Real-World Applications }

Separable ODEs model countless phenomena in science and engineering.

### Population Dynamics

**Exponential Growth:** $dP/dt = rP$
- Solution: $P(t) = P_0 e^{rt}$
- Models: bacteria, investments, viral videos

**Logistic Growth:** $dP/dt = rP(1 - P/K)$
- Solution: $P(t) = K/(1 + Ae^{-rt})$
- Models: species with limited resources, product adoption

### Physics

**Newton's Law of Cooling:** $dT/dt = k(T - T_{ambient})$
- Forensic science: time of death estimation
- Engineering: HVAC system design
- Daily life: beverage cooling

**Falling Objects with Air Resistance:** $dv/dt = g - kv$
- Terminal velocity: $v_{terminal} = g/k$
- Skydiving, raindrop dynamics

### Chemistry

**First-Order Reactions:** $dA/dt = -kA$
- Radioactive decay
- Drug metabolism
- Chemical kinetics

### Mixing Problems

Tank with inflow and outflow: $dy/dt = r_{in}c_{in} - r_{out}(y/V)$
- Water treatment plants
- Pharmaceutical manufacturing
- Environmental pollution modeling



\section{Examples}


\subsection{ Simple Exponential Growth }

The simplest separable ODE models exponential growth: dy/dx = y.
This can be separated as dy/y = dx, integrating gives ln|y| = x + C,
which simplifies to y = Ce^x. This models population growth, radioactive
decay (with negative constant), and compound interest.


\begin{lstlisting}
from mathhook import symbol, expr
from mathhook.ode import SeparableODESolver

x = symbol('x')
y = symbol('y')

# dy/dx = y (exponential growth)
rhs = y
solver = SeparableODESolver()

# General solution: y = C*e^x
solution = solver.solve(rhs, y, x)
print(f"General solution: {solution}")

# Particular solution with y(0) = 3
particular = solver.solve(rhs, y, x, initial=(0, 3))
print(f"Particular solution: {particular}")  # y = 3*e^x

\end{lstlisting}


\textbf{Output:}

\begin{verbatim}
y = 3*e^x
\end{verbatim}



\subsection{ Logistic Growth Model }

The logistic equation dy/dx = y(1-y) models population growth with carrying
capacity. Separating variables: dy/(y(1-y)) = dx. Using partial fractions:
(1/y + 1/(1-y))dy = dx. Integrating: ln|y/(1-y)| = x + C.
Solving for y: y = 1/(1 + Ce^(-x)). This S-curve is fundamental in ecology,
epidemiology, and marketing.


\begin{lstlisting}
from mathhook import symbol, expr
from mathhook.ode import SeparableODESolver

x, y = symbol('x'), symbol('y')

# dy/dx = y(1-y) (logistic growth)
rhs = y * (1 - y)
solver = SeparableODESolver()

# With y(0) = 0.1 (10% initial population)
solution = solver.solve(rhs, y, x, initial=(0, 0.1))
print(solution)  # y = 1/(1 + 9*e^(-x))

\end{lstlisting}


\textbf{Output:}

\begin{verbatim}
y = 1/(1 + 9*e^(-x))
\end{verbatim}



\subsection{ Newton's Law of Cooling }

Temperature change follows dT/dt = k(T - T_ambient). For T_ambient = 20°C,
this becomes dT/dt = k(T - 20). Separating: dT/(T-20) = k*dt.
Integrating: ln|T-20| = kt + C, so T = 20 + Ce^(kt). If T(0) = 100°C and
k = -0.05, then T = 20 + 80*e^(-0.05t). This models coffee cooling,
forensic time-of-death estimation, and HVAC systems.


\begin{lstlisting}
from mathhook import symbol, expr
from mathhook.ode import SeparableODESolver

t = symbol('t')
T = symbol('T')
k = -0.05

# dT/dt = k(T - 20)
rhs = k * (T - 20)
solver = SeparableODESolver()

# T(0) = 100 (coffee starts at 100°C)
solution = solver.solve(rhs, T, t, initial=(0, 100))
print(f"Temperature: {solution}")  # T = 20 + 80*e^(-0.05t)

# Evaluate at t = 10 minutes
temp_at_10 = solution.subs(t, 10)
print(f"After 10 min: {temp_at_10:.1f}°C")

\end{lstlisting}


\textbf{Output:}

\begin{verbatim}
T = 20 + 80*e^(-0.05t)
\end{verbatim}





\section{Conclusion}

Separable differential equations are the foundation of ODE solving. Their
simple form—factorable into $g(x)h(y)$—allows us to use basic integration
rather than advanced techniques. From exponential growth to Newton's cooling,
separable ODEs model fundamental processes in nature and engineering.

Key takeaways:
- **Recognition:** Can you factor the RHS as $g(x)h(y)$?
- **Method:** Separate variables, integrate both sides
- **Initial conditions:** Find the particular solution by substituting $y(x_0) = y_0$
- **Watch out:** Don't lose singular solutions when dividing by $h(y)$



\end{document}
