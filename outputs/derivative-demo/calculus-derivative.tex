\documentclass[11pt,a4paper]{article}
\usepackage{amsmath}
\usepackage{amssymb}
\usepackage{listings}
\usepackage{xcolor}
\usepackage{hyperref}

\lstset{
    language=Python,
    basicstyle=\ttfamily\small,
    keywordstyle=\color{blue},
    commentstyle=\color{gray},
    stringstyle=\color{red},
    showstringspaces=false,
    breaklines=true,
    frame=single,
    numbers=left,
    numberstyle=\tiny\color{gray}
}

\title{ Symbolic Differentiation }
\author{MathHook CAS}
\date{\today}

\begin{document}

\maketitle

\begin{abstract}
Computes the derivative of an expression with respect to a variable using
symbolic differentiation rules. Supports power rule, product rule, quotient
rule, chain rule, and derivatives of elementary functions.

\end{abstract}


\section{Mathematical Definition}

\begin{equation}
$$\frac{d}{dx} f(x) = \lim_{h \to 0} \frac{f(x+h) - f(x)}{h}$$

\end{equation}






\section{Examples}


\subsection{ Power Rule }

The derivative of x^n is n*x^(n-1). This is one of the fundamental
rules of calculus and applies to any real exponent n.


\begin{lstlisting}
x = symbol('x')
f = expr('x^3')
df = f.derivative(x)
# Result: 3*x^2

\end{lstlisting}


\textbf{Output:}

\begin{verbatim}
3*x^2
\end{verbatim}



\subsection{ Chain Rule }

When differentiating a composition of functions f(g(x)), we use the
chain rule: (f∘g)'(x) = f'(g(x)) · g'(x)


\begin{lstlisting}
x = symbol('x')
f = expr('sin(x^2)')
df = f.derivative(x)
# Result: 2*x*cos(x^2)

\end{lstlisting}


\textbf{Output:}

\begin{verbatim}
2*x*cos(x^2)
\end{verbatim}



\subsection{ Product Rule }

For the product of two functions u(x)·v(x), the derivative is
u'(x)·v(x) + u(x)·v'(x)


\begin{lstlisting}
x = symbol('x')
f = expr('x^2 * sin(x)')
df = f.derivative(x)
# Result: 2*x*sin(x) + x^2*cos(x)

\end{lstlisting}


\textbf{Output:}

\begin{verbatim}
2*x*sin(x) + x^2*cos(x)
\end{verbatim}






\end{document}
