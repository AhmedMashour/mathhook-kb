\documentclass[11pt,a4paper]{article}
\usepackage{amsmath}
\usepackage{amssymb}
\usepackage{listings}
\usepackage{xcolor}
\usepackage{hyperref}

\lstset{
    language=Python,
    basicstyle=\ttfamily\small,
    keywordstyle=\color{blue},
    commentstyle=\color{gray},
    stringstyle=\color{red},
    showstringspaces=false,
    breaklines=true,
    frame=single,
    numbers=left,
    numberstyle=\tiny\color{gray}
}

\title{ Node.js/TypeScript API Guide }
\author{MathHook CAS}
\date{\today}

\begin{document}

\maketitle

\begin{abstract}
Complete guide to using MathHook from Node.js and TypeScript via NAPI bindings.
Provides comprehensive documentation for the JavaScript/TypeScript API including installation,
quick start, API reference, integration patterns with Express/Next.js, and performance best practices.

\end{abstract}




\section{Introduction}

# Node.js/TypeScript API Guide

Complete guide to using MathHook from Node.js and TypeScript via NAPI bindings.

## Installation

```bash
npm install mathhook
# or
yarn add mathhook
# or
pnpm add mathhook
```

**Requirements**:
- Node.js 16.0 or higher
- npm 7.0 or higher

**Platform Support**:
- Linux (x86_64, aarch64)
- macOS (Intel, Apple Silicon)
- Windows (x86_64)

**TypeScript Support**: Type definitions included (.d.ts files bundled)

## Quick Start

### JavaScript

```javascript
const { Symbol, parse, simplify } = require('mathhook');

// Create symbols
const x = new Symbol('x');
const y = new Symbol('y');

// Build expressions
const expr = parse('x^2 + 2*x + 1');

// Simplify
const simplified = simplify(expr);
console.log(simplified.toString());  // (x + 1)^2
```

### TypeScript

```typescript
import { Symbol, Expression, parse, simplify } from 'mathhook';

// Create symbols (with type safety)
const x: Symbol = new Symbol('x');
const y: Symbol = new Symbol('y');

// Build expressions
const expr: Expression = parse('x^2 + 2*x + 1');

// Simplify
const simplified: Expression = simplify(expr);
console.log(simplified.toString());  // (x + 1)^2
```

## Why MathHook for Node.js?

### Performance Comparison

**Native Performance in JavaScript**:
- Rust core compiled to native code via NAPI
- No V8 overhead for mathematical operations
- Faster than pure JavaScript CAS libraries

```javascript
const { parse, simplify } = require('mathhook');

// Large polynomial expression
const terms = Array.from({length: 100}, (_, i) => `${i}*x^${i}`);
const exprStr = terms.join(' + ');

// MathHook
const start = Date.now();
const expr = parse(exprStr);
const result = simplify(expr);
const mathhookTime = Date.now() - start;

console.log(`MathHook: ${mathhookTime}ms`);
// Typical: MathHook 1-5ms vs JavaScript CAS 100-500ms
```

### When to Use MathHook

**Use MathHook when**:
- Building web applications with symbolic math (calculators, graphing, education)
- Server-side computation for math APIs
- Real-time symbolic computation requirements
- Need LaTeX parsing and rendering

**Integration Points**:
- Express/Fastify APIs for math endpoints
- Next.js/Nuxt.js server-side rendering
- WebSocket servers for interactive math applications
- GraphQL resolvers for mathematical queries





\section{Examples}


\subsection{ Basic Parsing and Simplification }

Parse mathematical expressions from strings and simplify them

\begin{lstlisting}
from mathhook import parse, simplify, expand

expr = parse('x + x')
result = simplify(expr)
print(result)  # 2*x

expr2 = parse('(x + 1)^2')
expanded = expand(expr2)
print(expanded)  # x^2 + 2*x + 1

\end{lstlisting}




\subsection{ TypeScript Type Safety }

Use TypeScript for type-safe mathematical operations

\begin{lstlisting}

\end{lstlisting}




\subsection{ Derivatives in TypeScript }

Compute symbolic derivatives with TypeScript type safety

\begin{lstlisting}
from mathhook import Symbol, parse, derivative

x = Symbol('x')
expr = parse('x^3')

# First derivative
df = derivative(expr, x)
print(df)  # 3*x^2

# Second derivative
d2f = derivative(expr, x, order=2)
print(d2f)  # 6*x

\end{lstlisting}




\subsection{ Express.js API Integration }

Build a REST API for mathematical operations using Express.js

\begin{lstlisting}

\end{lstlisting}




\subsection{ Next.js Server Actions }

Use MathHook in Next.js server actions for server-side computation

\begin{lstlisting}

\end{lstlisting}




\subsection{ WebSocket Server }

Build a WebSocket server for real-time mathematical computation

\begin{lstlisting}

\end{lstlisting}




\subsection{ Evaluation with Context }

Advanced evaluation with custom contexts and variable substitutions

\begin{lstlisting}
from mathhook import PyExpression as Expression, EvalContext

x = Expression.symbol("x")
y = Expression.symbol("y")

# Formula: x² + 2xy + y²
expr = x.pow(Expression.integer(2)).add(
    Expression.integer(2).multiply(x).multiply(y)
).add(y.pow(Expression.integer(2)))

# Create numerical context with variable substitutions
ctx = EvalContext.numeric({
    "x": Expression.integer(3),
    "y": Expression.integer(4)
})

# Evaluate: (3)² + 2(3)(4) + (4)² = 9 + 24 + 16 = 49
result = expr.evaluate_with_context(ctx)
print(result)  # 49

\end{lstlisting}







\end{document}
