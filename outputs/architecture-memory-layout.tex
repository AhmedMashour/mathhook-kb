\documentclass[11pt,a4paper]{article}
\usepackage{amsmath}
\usepackage{amssymb}
\usepackage{listings}
\usepackage{xcolor}
\usepackage{hyperref}

\lstset{
    language=Python,
    basicstyle=\ttfamily\small,
    keywordstyle=\color{blue},
    commentstyle=\color{gray},
    stringstyle=\color{red},
    showstringspaces=false,
    breaklines=true,
    frame=single,
    numbers=left,
    numberstyle=\tiny\color{gray}
}

\title{ Memory Layout Optimization }
\author{MathHook CAS}
\date{\today}

\begin{document}

\maketitle

\begin{abstract}
32-byte expression constraint and cache-line optimization strategy for maximum performance.
Covers memory layout details and performance implications.

\end{abstract}




\section{Introduction}

# Memory Layout Optimization

This chapter covers internal implementation details of memory layout optimization.

## 32-Byte Expression Constraint

Expressions are exactly 32 bytes to fit two expressions per 64-byte cache line.

### Cache-Line Optimization

Modern CPUs use 64-byte cache lines. MathHook's 32-byte expressions:

- Fit two expressions per cache line
- Minimize cache misses during traversal
- Improve CPU cache utilization
- Provide 10-100x speedup over Python-based systems

## Memory Layout Details

```rust
#[repr(C)]
pub struct Expression {
    // Total: 32 bytes
    discriminant: u8,     // 1 byte: expression type
    flags: u8,            // 1 byte: metadata flags
    _padding: [u8; 6],    // 6 bytes: alignment
    data: [u64; 3],       // 24 bytes: expression data
}
```

## Performance Implications

The 32-byte constraint:

- **Memory efficiency**: Compact representation
- **Cache efficiency**: Optimal cache-line usage
- **Traversal speed**: Fast tree traversal
- **Allocation speed**: Predictable allocation patterns





\section{Examples}





\end{document}
