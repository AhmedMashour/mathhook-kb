\documentclass[11pt,a4paper]{article}
\usepackage{amsmath}
\usepackage{amssymb}
\usepackage{listings}
\usepackage{xcolor}
\usepackage{hyperref}

\lstset{
    language=Python,
    basicstyle=\ttfamily\small,
    keywordstyle=\color{blue},
    commentstyle=\color{gray},
    stringstyle=\color{red},
    showstringspaces=false,
    breaklines=true,
    frame=single,
    numbers=left,
    numberstyle=\tiny\color{gray}
}

\title{ Mathematical Constants }
\author{MathHook CAS}
\date{\today}

\begin{document}

\maketitle

\begin{abstract}
MathHook provides built-in mathematical constants with exact symbolic representation
and high-precision numerical evaluation. Constants include π, e, i (imaginary unit),
golden ratio (φ), Euler-Mascheroni constant (γ), infinity, and undefined values.
All constants preserve mathematical exactness throughout symbolic computations.

\end{abstract}


\section{Mathematical Definition}

\begin{equation}
**Fundamental Constants:**

- **Pi**: $$\pi = 3.14159265358979323846\ldots$$ (ratio of circle's circumference to diameter)
- **Euler's Number**: $$e = 2.71828182845904523536\ldots = \lim_{n \to \infty} \left(1 + \frac{1}{n}\right)^n$$
- **Imaginary Unit**: $$i^2 = -1$$
- **Golden Ratio**: $$\varphi = \frac{1 + \sqrt{5}}{2} = 1.618033988749895\ldots$$
- **Euler-Mascheroni**: $$\gamma = \lim_{n \to \infty} \left(\sum_{k=1}^{n} \frac{1}{k} - \ln(n)\right) = 0.5772156649015329\ldots$$

\end{equation}






\section{Examples}


\subsection{ Using Pi in Expressions }

Pi constant for exact trigonometric and geometric calculations

\begin{lstlisting}
from mathhook import Expression, symbol, expr

pi = Expression.pi()
r = symbol('r')

# Circle area
area = expr('pi * r^2')

# Circumference
circumference = expr('2*pi*r')

# Exact trig values
assert expr('sin(pi)') == 0
assert expr('cos(pi)') == -1
assert expr('sin(pi/2)') == 1

\end{lstlisting}




\subsection{ Euler's Number (e) and Natural Logarithm }

Using e for exponential growth and logarithmic identities

\begin{lstlisting}
from mathhook import Expression, symbol, expr

e = Expression.e()
x = symbol('x')

# Exponential growth
growth = expr('e^(r*t)')

# Logarithm identities
assert expr('ln(e)') == 1
assert expr('ln(e^x)').simplify() == x
assert expr('e^(ln(x))').simplify() == x

# Derivative property
exp_x = expr('e^x')
assert exp_x.derivative(x) == exp_x

\end{lstlisting}




\subsection{ Imaginary Unit (i) and Euler's Identity }

Complex numbers and the most beautiful equation in mathematics

\begin{lstlisting}
from mathhook import Expression, expr

i = Expression.i()

# Complex numbers
z = expr('3 + 4*i')
magnitude = expr('sqrt(3^2 + 4^2)')
assert magnitude.simplify() == 5

# Euler's identity
euler = expr('e^(i*pi) + 1')
assert euler.simplify() == 0

# Powers of i
assert expr('i^2').simplify() == -1
assert expr('i^3').simplify() == expr('-i')
assert expr('i^4').simplify() == 1

\end{lstlisting}




\subsection{ Golden Ratio and Fibonacci Connection }

Golden ratio in geometry and its relationship to Fibonacci sequence

\begin{lstlisting}
from mathhook import Expression, expr

phi = Expression.golden_ratio()

# Exact form
phi_exact = expr('(1 + sqrt(5)) / 2')
assert phi.simplify() == phi_exact.simplify()

# Golden ratio property
assert expr('phi^2').simplify() == expr('phi + 1')

# Numerical value
# φ ≈ 1.618033988749895

\end{lstlisting}




\subsection{ Infinity and Undefined Values }

Working with unbounded limits and indeterminate forms

\begin{lstlisting}
from mathhook import Expression, symbol, expr

inf = Expression.infinity()

# Defined operations
assert expr('infinity + 1') == inf
assert expr('infinity * 2') == inf
assert expr('1 / infinity').simplify() == 0

# Limits
x = symbol('x')
limit = expr('lim(1/x, x, infinity)')
assert limit.simplify() == 0

# Indeterminate forms return Undefined

\end{lstlisting}




\subsection{ Real-World Physics: Harmonic Motion }

Using π and e in simple harmonic oscillator equations

\begin{lstlisting}
from mathhook import symbol, expr

# Harmonic oscillator
t = symbol('t')
position = expr('2 * cos(pi*t + pi/4)')
velocity = position.derivative(t)
acceleration = velocity.derivative(t)

# Verify differential equation
omega = expr('pi')
# a = -ω²x

\end{lstlisting}







\end{document}
